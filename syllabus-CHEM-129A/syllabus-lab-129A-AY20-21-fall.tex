\newpage

\hypertarget{sec:introduction}{%
\section{Introduction}\label{sec:introduction}}

Welcome to CHEM 129A, Organic Chemistry Laboratory! This course will
introduce you to one of the richest and most beautiful areas of modern
chemistry: \emph{chemistry of carbon compounds}. In CHEM 129A, we will
learn skills that are essential for performing experiments in organic
chemistry laboratory.

\emph{Read the syllabus carefully}. Almost all questions about the
course that you might ask can be answered by referencing the syllabus.
If you are uncertain that you understand all rules and regulations,
please contact the instructor.

\begin{itemize}
\tightlist
\item
  \textbf{Course name and number}: CHEM 129A (2 Units)
\item
  \textbf{Prerequisites}:CHEM 128A with a grade of C or better. CHEM
  128A can be taken concurrently.
\item
  \textbf{Instructor}: Hubert Muchalski, Ph.D.
\item
  \textbf{Instructor office}: Science 1 room 352
\item
  \textbf{Office Phone}: (559) 278-2711 (redirects to my cell phone)
\item
  \textbf{Email}{[}\^{}64ce62d3{]}: \url{hmuchalski@csufresno.edu} or
  \url{hmuchalski@mail.fresnostate.edu}
\item
  \textbf{Office Hours}: Mon/Wed 16:00--17:00.
\item
  \textbf{Course coordinator}: Dr.~Santanu Maitra; email:
  \url{smaitra@mail.fresnostate.edu}; office S-246; office hours: MW
  9:00--10:00
\end{itemize}

\newpage

\hypertarget{sec:course-goals}{%
\section{Course goals}\label{sec:course-goals}}

CHEM 129A is a two-unit introductory laboratory course in organic
chemistry. It is primarily concerned with introducing the tools and
techniques that organic chemists use to synthesize and investigate the
properties of organic compounds (see the \protect\hyperlink{slo}{list
below}). Some of these techniques are the same or similar to those you
learned in general chemistry courses but may be modified because the
experiments use very small amounts of material (micro-scale techniques).
Students who successfully complete CHEM 129A generally enroll in CHEM
129B, which further develops students' laboratory skills. Some students
then continue with CHEM 190, undergraduate research or independent
study.

\hypertarget{sec:slo}{%
\section{Student Learning Outcomes}\label{sec:slo}}

Students who successfully complete CHEM 129A should be able to:

\begin{itemize}
\tightlist
\item
  work safely in a laboratory;
\item
  communicate the structure and properties of organic molecules using
  common drawing and naming conventions;
\item
  analyze chemical structures and reaction conditions to predict the
  outcome chemical transformations;
\item
  maintain an accurate laboratory notebook that would allow a trained
  organic chemist to reproduce the results of an experiment;
\item
  plan an experiment by evaluating the information found in online
  databases and research articles;
\item
  carry out fundamental laboratory techniques such as extraction,
  crystallization, distillation, chromatography;
\item
  measure physical properties of organic compounds including melting and
  boiling points
\item
  assay purity and chemical identity of organic compounds using TLC, GC,
  and IR
\item
  communicate the results of experiments to the instructor and peers in
  a written form (lab report)
\end{itemize}

\hypertarget{sec:course-materials}{%
\section{Course materials}\label{sec:course-materials}}

\hypertarget{sec:textbook}{%
\subsection{Textbook}\label{sec:textbook}}

The face-to-face version of this course is based on the textbook
``Micro-scale Approach to Organic Laboratory Techniques'' by Pavia et
al.~Although this title is still listed as the official required course
textbook, we are in the process of phasing it out.

If you can obtain a copy of Pavia's textbook (5th or 6th edition) for
affordable price, you can use it in addition to the open access
materials listed below.

\begin{itemize}
\tightlist
\item
  \href{https://chem.libretexts.org/Bookshelves/Organic_Chemistry/Book:_Organic_Chemistry_Lab_Techniques_(Nichols)}{Organic
  Chemistry Lab Techniques (by Nichols)} {[}Web version{]}
\item
  \href{https://batch.libretexts.org/print/Letter/Finished/chem-93154/Full.pdf}{Organic
  Chemistry Lab Techniques (by Nichols)} {[}PDF, full text{]} -
  \emph{large download (350MB)}
\end{itemize}

\hypertarget{sec:course-materials-and-technology}{%
\subsection{Course materials and
technology}\label{sec:course-materials-and-technology}}

\begin{itemize}
\tightlist
\item
  \textbf{Personal protective equipment (PPE)}: Lab coat and approved
  safety goggles. Disposable nitrile gloves will be provided.
\item
  \textbf{Lab Notebook:} Large size exam Blue Book for each experiment.
\item
  \textbf{Document scanning tool}: A mobile device with a scanning app
  can convert paper documents to PDF files. There are number of options
  available for both iOS and Android. Find one that you like and learn
  how to use it.
\item
  \textbf{Canvas:} The central repository for all course materials and
  information is our Canvas site. The Canvas site will house your
  grades, links to handouts, videos, and other materials.
\item
  \textbf{Zoom:} Virtual office hours meetings will be held via Zoom.
  Links and passwords to zoom meetings will be provided by the
  instructor.
\item
  \textbf{Office 365 Apps}: Many assignments in this course are
  submitted as Word documents and are based on tamplates that are Word
  documents. Office 365 apps are available for Fresno State students.
  The suite includes the scanning app \emph{Office Lens}. Learn more
  here:
  \url{https://help.fresnostate.edu/students/software/office365.php}.
\end{itemize}

\hypertarget{sec:laboratory-code-of-safe-practices}{%
\section{Laboratory Code of Safe
Practices}\label{sec:laboratory-code-of-safe-practices}}

\begin{enumerate}
\def\labelenumi{\arabic{enumi}.}
\tightlist
\item
  NO food or drink in the laboratory.
\item
  Wear clothing appropriate for laboratory work.
\item
  Select and correctly use appropriate Personal Protective Equipment
  (PPE).
\item
  Know what to do and who to contact in an emergency in the laboratory.
\item
  Avoid distractions and be alert to and aware of your surroundings and
  potential hazards in your area.
\item
  Maintain a safe and clean work area.
\item
  Only conduct experiments or procedures approved by your lab instructor
  or research advisor.
\item
  Understand the common chemical hazards and hazards specific to the
  chemicals and procedures with which you are working.
\item
  Understand and follow best practices on how to handle, transport,
  store, and dispose of chemicals safely.
\item
  If any equipment, glassware, or procedures are not working properly or
  as expected, notify your instructor before proceeding.
\item
  Notify your instructor if you have, develop, or may develop any
  medical conditions (e.g.~severe asthma, limited mobility, vision
  impairment, pregnancy, etc) that may affect your safety in the
  laboratory or sensitivity to chemicals, so that your instructor can
  properly advise or accommodate you on minimizing the risks associated
  with laboratory work.
\end{enumerate}

These principles will be discussed in detail during the first week of
class. More information can be found here: \url{https://goo.gl/1UFRbo}.
Also, refer to
\href{https://www.acs.org/content/dam/acsorg/about/governance/committees/chemicalsafety/publications/acs-safety-guidelines-academic.pdf}{\emph{Guidelines
for Chemical Laboratory Safety in Academic Institutions}} published by
American Chemical Society.

\hypertarget{sec:expectations}{%
\section{Expectations}\label{sec:expectations}}

Due to the ongoing coronavirus pandemic this class is delivered
virtually in a blended mode, as a combination of synchronous and
asynchronous activities. Activities planned for synchronous meetings
will rely on breakout rooms. Although the sessions may be recorded,
activities within breakout rooms are not being recorded.

Students are expected to be present during virtual class sessions
equipped with a basic understanding of the concepts that will allow you
to fully participate in discussions. Preparing for the class means
making honest effort to learn on your own. This is done by:

\begin{itemize}
\tightlist
\item
  reading the assigned material related to the experiment;
\item
  preparing the pre-lab assignment;
\item
  reviewing the relevant techniques for each experiment (listed in the
  introduction section for each experiment);
\item
  watching videos demonstrating laboratory techniques; and
\item
  reviewing safety data sheets (SDS) for chemicals you will be working
  with.
\end{itemize}

\hypertarget{sec:types-of-graded-work}{%
\section{Types of graded work}\label{sec:types-of-graded-work}}

\hypertarget{sec:pre-lab}{%
\subsection{Pre-lab}\label{sec:pre-lab}}

Before carrying out the experiment you will complete a pre-lab
assignment to prepare for the scheduled experiment. Pre-lab assignment
may include assigned reading, short quiz, pre-experiment documentation
(reagent table, risk assessment).

\hypertarget{sec:experiments}{%
\subsection{Experiments}\label{sec:experiments}}

Majority of the classroom time will be spent on carrying out
experiments, making observations, and generating results. Although
results and observations are not graded directly, they are important
components of post-lab assignments. For example, not making a key
observation or obtaining poor quality data may reduce your grade on the
experiment.

\hypertarget{sec:lab-notebook}{%
\subsection{Lab notebook}\label{sec:lab-notebook}}

Assume that you're in note-taking mode at all time when you are in the
lab. Well prepared pre-lab and in-lab notes are very important because
they are the basis for the post-lab summary or report that you will be
asked to complete.

In-lab notes document what you did in the lab (including notes taken
during lecture and discussions) and the results of your experimental
work. To know what to write and not to write in the notebook is a
balancing act of relevance and brevity. You need write down details that
are relevant and necessary for you (or someone else) to reproduce the
experiment. Assume that your notes are going to be read by a trained
organic chemist. For example, you don't have to explain what is a
round-bottomed flask but you should mention the size of flask, joint
size, etd.

Keep all prints (IR and NMR spectra, gas chromatograms, etc.) and
sketches of TLC plates as part of you notes portfolio. Your lab notebook
is subject to evaluation at any time and must be kept current.
Instructions on how to keep laboratory notebook are discussed in the
textbook and will be reviewed in the lab. You will submit your in-lab
notes for review before you leave the lab room.

\hypertarget{sec:post-lab}{%
\subsection{Post-lab}\label{sec:post-lab}}

Post-lab assignments conclude each experiment. Post-lab will be in a
form of a short write-up, formal typed report, or problem set. Details
for each post-lab assignment will be posted on Canvas. Document
templates will be available for formal lab reports.

\hypertarget{sec:exams-and-quizzes}{%
\subsection{Exams and Quizzes}\label{sec:exams-and-quizzes}}

Quizzes are short assessments (often multiple-choice) that check your
knowledge before the class (based on reading or videos assigned). Exams
are longer assessments based on the already covered material and will be
take-home assignments

\hypertarget{sec:grading}{%
\section{Grading}\label{sec:grading}}

Graded assignments are organized into assignment groups (see
Table~\ref{tbl:assignments}) and each group contributes to the final
grade.

Points earned for assignments in different categories are not equivalent
and should not be treated as such. For example, it may take more work
and effort to earn 50 poins for a report than 100 points for completing
a simulation.

\hypertarget{tbl:assignments}{}
\begin{longtable}[]{@{}ll@{}}
\caption{\label{tbl:assignments}Assignment groups and their
weights}\tabularnewline
\toprule
Assignment Group & \\
\midrule
\endfirsthead
\toprule
Assignment Group & \\
\midrule
\endhead
Chemical Safety & 5\% \\
Language of organic chemistry & 5\% \\
Solubility & 6\% \\
Crystallization & 6\% \\
Extraction & 6\% \\
Chromatography & 6\% \\
Distillation & 6\% \\
Analytical techniques (GC, IR) & 6\% \\
Synthesis of acetaminophen & 6\% \\
Synthesis of isoamyl acetate & 6\% \\
Nucleophilic substitution & 6\% \\
Grignard synthesis of benzoic acid & 6\% \\
Take-home exams (4) & 20\% \\
Practicals (2) & 10\% \\
Perusall reading (extra credit) & 3\% \\
\textbf{TOTAL} & 103\% \\
\bottomrule
\end{longtable}

Final grade will be determined based on overall performance according to
the table below (Table~\ref{tbl:letter-grades}).

\hypertarget{tbl:letter-grades}{}
\begin{longtable}[]{@{}ll@{}}
\caption{\label{tbl:letter-grades}Final letter grade
brackets}\tabularnewline
\toprule
Grade & Total Score \\
\midrule
\endfirsthead
\toprule
Grade & Total Score \\
\midrule
\endhead
A & 90--100\% \\
B & 80--89\% \\
C & 70--79\% \\
D & 60-69\% \\
F & \textless60\% \\
\bottomrule
\end{longtable}

\hypertarget{sec:course-policies}{%
\section{Course policies}\label{sec:course-policies}}

\hypertarget{sec:attendance-participation-and-late-workmake-up-policy}{%
\subsection{Attendance, participation, and late work/make-up
policy}\label{sec:attendance-participation-and-late-workmake-up-policy}}

Students are expected to attend and actively participate in all virtual
class sessions. The course attendance policy follows university's
\href{http://www.fresnostate.edu/academics/facultyaffairs/documents/apm/232.pdf}{APM
232: Policy on Student Absence}.

Late assignments will not be accepted for a grade unless the absence
meets the guidelines set forth by the univeristy policy
\href{http://www.fresnostate.edu/academics/facultyaffairs/documents/apm/232.pdf}{APM
232: Policy on Student Absence}

\hypertarget{sec:technology-issues-when-submitting-work}{%
\subsection{Technology issues when submitting
work}\label{sec:technology-issues-when-submitting-work}}

For assignments submitted electronically, it is your responsibility to
make sure they are submitted on time, through any means necessary, even
if technology issues arise. If a tech issue arises, it is your
responsibility to find another way to get it to the instructor (for
example, via an email attachment). Technology issues that are avoidable
or resolved with a simple work-around will not be considered valid
grounds for a deadline extension. For example, if you are trying to
upload a Lab to Canvas and Canvas won't accept the file, you should try
again later or send the file as an email attachment until you can upload
it successfully.

\hypertarget{sec:academic-dishonesty}{%
\subsection{Academic Dishonesty}\label{sec:academic-dishonesty}}

For most assignments you are allowed and encouraged to work with others.
However, the final product that you submit for feedback must be the
result of your own efforts. Therefore you may share ideas and strategies
with others, but collaboration on the actual finished product you submit
is not allowed. Your work is expected to be the product of your own
thinking, written and explained in your own words with no parts of the
work copied from external sources such as books or websites, and done
clearly enough in your own mind that you could explain the work from
start to finish if asked. Specifically, this excludes:

\begin{itemize}
\tightlist
\item
  copying work from another student;
\item
  copying work from a website;
\item
  paraphrasing work done by another student or from print or internet
  resources---i.e.~putting it in your own words---without coming up with
  the main ideas and strategies yourself; and
\item
  \emph{allowing or enabling} another student to copy or paraphrase work
  that you did, even if you did the original work yourself.
\end{itemize}

Violation of this policy is considered ``academic dishonesty'' and
carries with it strong punitive measures mandated by Fresno State,
including possible automatic failure of the course or suspension from
the university. For details, please see APM 235 by going to
\url{http://www.fresnostate.edu/aps/documents/apm/235.pdf}.

You may feel tempted to academic dishonesty at some point in the
semester. The work can be difficult, and many of you are under a lot of
stress. If you are considering academic dishonesty, please STOP, take a
breath, and remember that your classmates and I want you to succeed in
the course. You are not alone, and you have a strong network in the
class for getting help. The revision and resubmission policies mean that
it's OK to turn in work that isn't perfect. There is no need to be
academically dishonest! Just do your best on the work, and you'll have
the chance to revise it later.

\hypertarget{sec:dropping-the-course-after-the-census-date}{%
\subsection{Dropping the course after the census
date}\label{sec:dropping-the-course-after-the-census-date}}

A \emph{serious and compelling reason} is defined as an unexpected
condition that is not present prior to enrollment in the course that
unexpectedly arises and interferes with a student's ability to attend
class meetings and/or complete course requirements. The reason must be
acceptable to and verified by the instructor of record and the
department chair. The condition must be stated in writing on the
appropriate form. The student must provide documentation that
substantiates the condition.

Failing or performing poorly in a class is not an acceptable ``serious
and compelling reason'' within the University policy, nor is
dissatisfaction with the subject matter, class or instructor.

\hypertarget{sec:university-policies-and-disclaimers}{%
\section{University policies and
disclaimers}\label{sec:university-policies-and-disclaimers}}

\hypertarget{sec:vaccinations}{%
\subsection{Vaccinations}\label{sec:vaccinations}}

In order to create a safe environment on campus, all students must be
vaccinated against the SARS-CoV-02 virus, or obtain an exemption, in
order to attend classes on campus or access any services on campus.
Documentation of the first dose of the vaccination must be uploaded to
the student portal by Aug 20, and documentation of the final dose by
Sept 30. Students may request an exemption to the vaccine requirement by
going to their student portal to complete the COVID self-certification.
Students with vaccination exemptions are subject to weekly COVID
testing.

A person is not allowed to come to campus if the person is not been
vaccinated, and has not been granted an exemption, or the person has
been granted an exemption, but has not tested negative through required
weekly testing.

\hypertarget{sec:health-screening}{%
\subsection{Health Screening}\label{sec:health-screening}}

Students who come to campus and/or are participating in off-campus
in-person experiential learning will be required to complete a daily
health screening before coming to campus or learning site.

A person is not allowed to come to campus if the person is experiencing
COVID-19 symptoms (vaccinated or not); or the person has tested positive
within the past 10 days.

If you have a suspected or confirmed case of COVID-19, please complete
the campus online reporting form. A campus official will reply to
provide guidance and information.

\hypertarget{sec:safety-measures}{%
\subsection{Safety Measures}\label{sec:safety-measures}}

Face coverings are required to be worn indoors on-campus and during
in-person classes (vaccinated or not), and/or in accordance with
learning site requirements if participating in off-campus experiential
learning, to reduce the risk of community spread of COVID-19. The
Student Health and Counseling Center has complimentary masks available
for students who need them. The mask requirement may be modified if/when
transmission rates in Fresno Country drop below the threshold identified
by the CDC.

\hypertarget{sec:students-with-disabilities}{%
\subsection{Students with
disabilities}\label{sec:students-with-disabilities}}

Upon identifying themselves to the instructor and the university,
students with disabilities will receive reasonable accommodation for
learning and evaluation. For more information, contact Services to
Students with Disabilities in the Henry Madden Library, Room 1202
(278-2811).

\hypertarget{sec:links-to-other-important-policies}{%
\subsection{Links to other important
policies}\label{sec:links-to-other-important-policies}}

\begin{itemize}
\tightlist
\item
  Class Schedule Policies:
  \url{http://fresnostate.edu/studentaffairs/classschedule/policy/}
\item
  Copyright Policy: \url{http://libguides.csufresno.edu/copyright}
\item
  Students with Disabilities:
  \url{http://fresnostate.edu/studentaffairs/careers/students/interests/disabilities.html}
\item
  Academic Integrity and Honor Code:
  \url{http://www.fresnostate.edu/academics/facultyaffairs/documents/apm/236.pdf}
\item
  Policy on Cheating and Plagiarism:
  \url{http://fresnostate.edu/studentaffairs/studentconduct/policies/cheating-plagiarism.html}
\item
  Add/Drop Course:
  \url{http://www.fresnostate.edu/studentaffairs/registrar/registration/}
\item
  Computer requirements:
  \url{https://www.fresnostate.edu/catalog/academic-regulations/index.html\#computerreq}
\item
  Disruptive classroom behavior:
  \url{http://www.fresnostate.edu/academics/facultyaffairs/documents/apm/419.pdf}
\end{itemize}

\hypertarget{sec:university-services}{%
\section{University Services}\label{sec:university-services}}

\begin{itemize}
\tightlist
\item
  \href{http://fresnostateasi.org/}{Associated Students, Inc.}
\item
  \href{http://fresnostate.edu/studentaffairs/dsc/index.html}{Dream
  Success Center}
\item
  \href{http://fresnostate.edu/studentaffairs/lrc}{Learning Center
  Information}
\item
  \href{https://www.fresnostate.edu/studentaffairs/health/}{Student
  Health and Counseling Center}
\item
  \href{http://www.fresnostate.edu/artshum/writingcenter/}{Writing
  Center}
\end{itemize}
