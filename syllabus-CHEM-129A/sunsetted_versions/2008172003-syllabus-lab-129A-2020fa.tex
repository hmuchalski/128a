\hypertarget{virtual-classroom-policies-and-statements}{%
\section{Virtual classroom policies and
statements}\label{virtual-classroom-policies-and-statements}}

\hypertarget{intellectual-property}{%
\subsection{Intellectual Property}\label{intellectual-property}}

As part of your participation in virtual/online instruction, please
remember that the same student conduct rules that are used for in-person
classrooms instruction also apply for virtual/online classrooms.
Students are prohibited from any unauthorized recording, dissemination,
or publication of any academic presentation, including any online
classroom instruction, for any commercial purpose. In addition, students
may not record or use virtual/online instruction in any manner that
would violate copyright law. Students are to use all online/virtual
instruction exclusively for the educational purpose of the online class
in which the instruction is being provided. Students may not re-record
any online recordings or post any online recordings on any other format
(e.g., electronic, video, social media, audio recording, web page,
internet, hard paper copy, etc.) for any purpose without the explicit
written permission of the faculty member providing the instruction.
Exceptions for disability-related accommodations will be addressed by
Services for Students with Disabilities (SSD) working in conjunction
with the student and faculty member.

In addition to course policies, you are expected to be familiar with
Academic Regulations described in the
\href{http://www.fresnostate.edu/catalog/academic-regulations/}{University
Catalog} as well as policies listed below.

\hypertarget{health-screening}{%
\subsection{Health screening}\label{health-screening}}

Students who come to campus for face-to-face classes will be required to
complete a daily health screening which will include temperature checks.
If you have experienced COVID-19 symptoms and/or have tested positive
within the past 10 days; or if you have had close contact (less than 6
feet for longer than 15 minutes while unmasked) with a suspected or
confirmed COVID-19 patient within the past 14 days, you are not allowed
to come to campus. Please complete the campus
\href{https://fresnostate.co1.qualtrics.com/jfe/form/SV_3faIAsuC8CzuFjD?Q_FormSessionID=FS_UFJ902LXgDJbKeZ}{online
reporting form}. A campus official will reply to provide guidance and
information.

\hypertarget{safety-measures}{%
\subsection{Safety Measures}\label{safety-measures}}

Consistent with the Governor's order and updated state public-health
guidelines, face masks or cloth face coverings are required to be worn
in public spaces on-campus and during in-person classes to reduce
possible exposure to COVID-19 and prevent the spread of the virus.
Physical distancing must be practiced by maintaining 6 feet of distance
between individuals. Good hygiene of hand washing for a minimum of 20
seconds or using hand sanitizer is required. Please avoid touching your
face with unclean hands. Disposable face masks will be provided to
anyone who arrives to campus without one. Please see university website
for the most updated information:
\href{http://www.fresnostate.edu/president/coronavirus/index.html}{www.fresnostate.edu/coronavirus}

\hypertarget{contact-information-for-course-coordinator-department-chair-and-college-dean}{%
\subsection{Contact information for course coordinator, department chair
and college
dean}\label{contact-information-for-course-coordinator-department-chair-and-college-dean}}

If there are questions or concerns that you have about this course that
you and I are not able to resolve, please feel free to contact the
course coordinator Dr.~Hubert Muchalski. The the issue cannot be
resolved with the lab coordinator, please contact the department and/or
college administrators:

\begin{itemize}
\tightlist
\item
  Dr.~Joy Goto (Chair, Chemistry Department): jgoto@mail.fresnostate.edu
\item
  Chemistry Department Office: (559) 278-2103
\item
  Dr.~Christopher Meyer (Dean, College of Science and Mathematics):
  cmeyer@mail.fresnostate.edu
\end{itemize}

\hypertarget{changelog}{%
\section{Changelog}\label{changelog}}

This syllabus and schedule are subject to change in the event of
extenuating circumstances. If you are absent from class, it is your
responsibility to check on announcements made while you were absent.
Changes and corrections are listed in the changelog below and will be
announced on Canvas.

\begin{itemize}
\tightlist
\item
  2020-08-17: Published on Canvas; added COVID-19 policies
\item
  2020-08-11: Draft sent for to instructors for review
\end{itemize}

\newpage

\hypertarget{introduction}{%
\section{Introduction}\label{introduction}}

Welcome to CHEM 129A, Organic Chemistry Laboratory! This course will
introduce you to one of the richest and most beautiful areas of modern
chemistry: \emph{chemistry of carbon compounds}. In CHEM 129A, we will
learn skills that are essential for performing experiments in organic
chemistry laboratory.

\emph{Read the syllabus carefully}. Almost all questions about the
course that you might ask can be answered by referencing the syllabus.
If you are uncertain that you understand all rules and regulations,
please contact the instructor. Also, the syllabus for my sections
differs slightly from those used in other sections

\begin{itemize}
\tightlist
\item
  \textbf{Course name and number}: CHEM 129A (2 Units)
\item
  \textbf{Prerequisites}:CHEM 128A with a grade of C or better. CHEM
  128A can be taken concurrently.
\item
  \textbf{Instructor}: Hubert Muchalski, PhD
\item
  \textbf{Contact}: phone (559) 278-2711, or email
  \url{hmuchalski@mail.fresnostate.edu}
\item
  \textbf{Office Hours}: Appointments can be scheduled through calendar
  function ``Find Appointments'' on Canvas.
\end{itemize}

\hypertarget{what-is-chem-129a}{%
\section{What is CHEM 129A}\label{what-is-chem-129a}}

CHEM 129A is a two-unit introductory laboratory course in organic
chemistry. It is primarily concerned with introducing the tools and
techniques that chemists use to synthesize and investigate the
properties of organic compounds (see the \protect\hyperlink{slo}{list
below}). Some of these techniques are the same or similar to those you
learned in general chemistry courses but may be modified because the
experiments use very small amounts of material (micro-scale techniques).
Students who successfully complete CHEM 129A generally enroll in CHEM
129B, which further develops students' laboratory skills. Some students
then continue with CHEM 190, undergraduate research or independent
study.

\hypertarget{slo}{%
\subsection{Student Learning Outcomes}\label{slo}}

Students who successfully complete CHEM 129A should be able to:

\begin{itemize}
\tightlist
\item
  understand how to work safely in the laboratory, including the
  disposal of chemical wastes
\item
  communicate the structure and properties of organic molecules using
  common drawing and naming conventions
\item
  analyze chemical structures and reaction conditions to make and defend
  predictions about chemical transformations
\item
  maintain an accurate laboratory notebook that would allow a trained
  organic chemist to repeat the experimental and reproduce the results
\item
  use drawing software such as ChemDraw to draw chemical structures and
  reactions
\item
  plan an experiment by evaluating the information found in online
  databases and research articles
\item
  conduct experiments that involve laboratory techniques such as
  extraction, crystallization, distillation, and chromatography
\item
  measure physical properties of organic compounds including melting and
  boiling points
\item
  analyze purity and chemical identity of organic compounds using TLC,
  GC, and IR
\item
  analyze the results of an experiment and be able to identify sources
  of error and suggest improvements
\item
  communicate the results of experiments to the instructor and peers in
  a written form (lab report)
\end{itemize}

\hypertarget{course-materials-and-technology}{%
\section{Course materials and
technology}\label{course-materials-and-technology}}

\begin{itemize}
\tightlist
\item
  \textbf{Textbook:} ``A Micro-scale Approach to Organic Laboratory
  Techniques'' by Donald Pavia et al.~published by Thompson/Brooks Cole.
  All page/section/experiment numbers refer to the 6th edition. Previous
  editions will also be sufficient to learn the material, but
  page/section/experiment numbers may differ.
\item
  \textbf{Canvas:} The central repository for all course materials and
  information is our Canvas site, accessible through
  \url{https://fresnostate.instructure.com/courses/3782}. The Canvas
  site will house your grades, links to handouts, videos, and other
  materials.
\item
  \textbf{Zoom:} Virtual class meetings will be held via Zoom. Links and
  passwords to zoom meetings will be published on Canvas.
\item
  \textbf{Document scanning}: Many assignments in this course are
  designed to be prepared by hand on paper. Few people own document
  scanners nowadays, but a mobile device with a scanning app can do a
  sufficient job at converting paper documents into PDFs. There are
  number of options available for both iOS and Android. Find one that
  you like and learn how to use it.
\end{itemize}

\hypertarget{laboratory-code-of-safe-practices}{%
\section{Laboratory Code of Safe
Practices}\label{laboratory-code-of-safe-practices}}

\begin{enumerate}
\def\labelenumi{\arabic{enumi}.}
\tightlist
\item
  NO food or drink in the laboratory.
\item
  Wear clothing appropriate for laboratory work.
\item
  Select and correctly use appropriate Personal Protective Equipment
  (PPE).
\item
  Know what to do and who to contact in an emergency in the laboratory.
\item
  Avoid distractions and be alert to and aware of your surroundings and
  potential hazards in your area.
\item
  Maintain a safe and clean work area.
\item
  Only conduct experiments or procedures approved by your lab instructor
  or research advisor.
\item
  Understand the common chemical hazards and hazards specific to the
  chemicals and procedures with which you are working.
\item
  Understand and follow best practices on how to handle, transport,
  store, and dispose of chemicals safely.
\item
  If any equipment, glassware, or procedures are not working properly or
  as expected, notify your instructor before proceeding.
\item
  Notify your instructor if you have, develop, or may develop any
  medical conditions (e.g.~severe asthma, limited mobility, vision
  impairment, pregnancy, etc) that may affect your safety in the
  laboratory or sensitivity to chemicals, so that your instructor can
  properly advise or accommodate you on minimizing the risks associated
  with laboratory work.
\end{enumerate}

These principles will be discussed in detail during the first week of
class. More information can be found here: \url{https://goo.gl/1UFRbo}.
Also, refer to
\href{https://www.acs.org/content/dam/acsorg/about/governance/committees/chemicalsafety/publications/acs-safety-guidelines-academic.pdf}{\emph{Guidelines
for Chemical Laboratory Safety in Academic Institutions}} published by
American Chemical Society.

\hypertarget{expectations}{%
\section{Expectations}\label{expectations}}

Due to the ongoing coronavirus pandemic this class is delivered
virtually in a blended mode, as a combination of synchronous and
asynchronous activities. Activities planned for synchronous meetings
will rely on breakout rooms. Although the sessions may be recorded,
activities within breakout rooms are not being recorded.

I expect that you will be present during virtual class sessions equipped
with a basic understanding of the concepts that will allow you to fully
participate in discussions. Preparing for the class means making honest
effort to learn on your own. This is done by:

\begin{itemize}
\tightlist
\item
  reading the assigned material related to the experiment;
\item
  preparing the pre-lab assignment;
\item
  reviewing the relevant techniques for each experiment (listed in the
  introduction section for each experiment);
\item
  watching videos demonstrating laboratory techniques; and
\item
  reviewing safety data sheets (SDS) for chemicals you will be working
  with.
\end{itemize}

\hypertarget{grading}{%
\section{Grading}\label{grading}}

\begin{longtable}[]{@{}ll@{}}
\toprule
Grade & Total Score\tabularnewline
\midrule
\endhead
A & 90--100\%\tabularnewline
B & 80--89\%\tabularnewline
C & 70--79\%\tabularnewline
D & 60-69\%\tabularnewline
F & \textless60\%\tabularnewline
\bottomrule
\end{longtable}

Graded assignments are organized into 11 assignment groups. Ten
content-driven groups contribute 8\% of the final grade. Four exams
contribute the remaining 20\%.

\begin{longtable}[]{@{}ll@{}}
\toprule
Module & \%Weight\tabularnewline
\midrule
\endhead
Solubility & 8\%\tabularnewline
Crystallization & 8\%\tabularnewline
Extraction & 8\%\tabularnewline
Chromatography & 8\%\tabularnewline
Distillation \& GC & 8\%\tabularnewline
Fischer Esterification & 8\%\tabularnewline
Nucleophilic Substitution & 8\%\tabularnewline
Synthesis of Acetaminophen & 8\%\tabularnewline
Grignard Synthesis of Benzoic Acid & 8\%\tabularnewline
Miscellaneous & 8\%\tabularnewline
Exams (5\% each) & 20\%\tabularnewline
Total & 100\%\tabularnewline
\bottomrule
\end{longtable}

\hypertarget{course-policies}{%
\section{Course policies}\label{course-policies}}

\hypertarget{technology-issues-when-submitting-work}{%
\subsection{Technology issues when submitting
work}\label{technology-issues-when-submitting-work}}

For assignments submitted electronically, it is your responsibility to
make sure they are submitted on time, through any means necessary, even
if technology issues arise. If a tech issue arises, it is your
responsibility to find another way to get it to me (for example, via an
email attachment). Technology issues that are avoidable or resolved with
a simple work-around will not be considered valid grounds for a deadline
extension. For example, if you are trying to upload a Lab to Canvas and
Canvas won't accept the file, you should try again later or send the
file as an email attachment until you can upload it successfully.

\hypertarget{respondus-lockdown-browser}{%
\subsection{Respondus Lockdown
Browser}\label{respondus-lockdown-browser}}

Respondus Lockdown Browser is a custom browser that locks down the
testing environment within Canvas. When students use the Respondus
Lockdown Browser they are unable to print, copy, go to a URL, or access
other applications. When an assessment is started, students are locked
into it until they submit it for grading. Available for both Windows and
Mac. Respondus Lockdown Browser does not work on a Chromebook.

Respondus Lockdown Browser uses a standard Windows or Mac installer that
can be downloaded by faculty or students from the following link (note:
this link is unique for Fresno State):

\url{http://www.respondus.com/lockdown/download.php?id=749643058}

\hypertarget{academic-dishonesty}{%
\subsection{Academic Dishonesty}\label{academic-dishonesty}}

For most assignments you are allowed and encouraged to work with others.
However, the final product that you submit for feedback must be the
result of your own efforts. Therefore you may share ideas and strategies
with others, but collaboration on the actual finished product you submit
is not allowed. Your work is expected to be the product of your own
thinking, written and explained in your own words with no parts of the
work copied from external sources such as books or websites, and done
clearly enough in your own mind that you could explain the work from
start to finish if asked. Specifically, this excludes:

\begin{itemize}
\tightlist
\item
  copying work from another student;
\item
  copying work from a website;
\item
  paraphrasing work done by another student or from print or internet
  resources---i.e.~putting it in your own words---without coming up with
  the main ideas and strategies yourself; and
\item
  \emph{allowing or enabling} another student to copy or paraphrase work
  that you did, even if you did the original work yourself.
\end{itemize}

Violation of this policy is considered ``academic dishonesty'' and
carries with it strong punitive measures mandated by Fresno State,
including possible automatic failure of the course or suspension from
the university. For details, please see APM 235 by going to
\url{http://www.fresnostate.edu/aps/documents/apm/235.pdf}.

You may feel tempted to academic dishonesty at some point in the
semester. The work can be difficult, and many of you are under a lot of
stress. If you are considering academic dishonesty, please STOP, take a
breath, and remember that your classmates and I want you to succeed in
the course. You are not alone, and you have a strong network in the
class for getting help. The revision and resubmission policies mean that
it's OK to turn in work that isn't perfect. There is no need to be
academically dishonest! Just do your best on the work, and you'll have
the chance to revise it later.

\hypertarget{dropping-the-course-after-the-census-date}{%
\subsection{Dropping the course after the census
date}\label{dropping-the-course-after-the-census-date}}

A \emph{serious and compelling reason} is defined as an unexpected
condition that is not present prior to enrollment in the course that
unexpectedly arises and interferes with a student's ability to attend
class meetings and/or complete course requirements. The reason must be
acceptable to and verified by the instructor of record and the
department chair. The condition must be stated in writing on the
appropriate form. The student must provide documentation that
substantiates the condition.

Failing or performing poorly in a class is not an acceptable ``serious
and compelling reason'' within the University policy, nor is
dissatisfaction with the subject matter, class or instructor.

\hypertarget{university-policies-and-disclaimers}{%
\section{University policies and
disclaimers}\label{university-policies-and-disclaimers}}

\textbf{Students with Disabilities}: Upon identifying themselves to the
instructor and the university, students with disabilities will receive
reasonable accommodation for learning and evaluation. For more
information, contact Services to Students with Disabilities in the Henry
Madden Library, Room 1202 (278-2811).

\begin{itemize}
\tightlist
\item
  Class Schedule Policies:
  \url{http://fresnostate.edu/studentaffairs/classschedule/policy/}
\item
  Copyright Policy: \url{http://libguides.csufresno.edu/copyright}
\item
  Students with Disabilities:
  \url{http://fresnostate.edu/studentaffairs/careers/students/interests/disabilities.html}
\item
  Academic Integrity and Honor Code:
  \url{http://www.fresnostate.edu/academics/facultyaffairs/documents/apm/236.pdf}
\item
  Policy on Cheating and Plagiarism:
  \url{http://fresnostate.edu/studentaffairs/studentconduct/policies/cheating-plagiarism.html}
\item
  Add/Drop Course:
  \url{http://www.fresnostate.edu/studentaffairs/registrar/registration/}
\item
  Computer requirements:
  \url{https://www.fresnostate.edu/catalog/academic-regulations/index.html\#computerreq}
\item
  Disruptive classroom behavior:
  \url{http://www.fresnostate.edu/academics/facultyaffairs/documents/apm/419.pdf}
\end{itemize}

\hypertarget{university-services}{%
\section{University Services}\label{university-services}}

\begin{itemize}
\tightlist
\item
  \href{http://fresnostateasi.org/}{Associated Students, Inc.}
\item
  \href{http://fresnostate.edu/studentaffairs/dsc/index.html}{Dream
  Success Center}
\item
  \href{http://fresnostate.edu/studentaffairs/lrc}{Learning Center
  Information}
\item
  \href{https://www.fresnostate.edu/studentaffairs/health/}{Student
  Health and Counseling Center}
\item
  \href{http://www.fresnostate.edu/artshum/writingcenter/}{Writing
  Center}
\end{itemize}
