\newpage

Welcome to CHEM 129A, Organic Chemistry Laboratory! This course will
introduce you to one of the richest and most beautiful areas of modern
chemistry: \emph{chemistry of carbon compounds}. In CHEM 129A, we will
learn skills that are essential for performing experiments in organic
chemistry laboratory.

\emph{Read the syllabus carefully}. Almost all questions about the
course that you might ask can be answered by referencing the syllabus.
If you are uncertain that you understand all rules and regulations,
please contact me. Also, the syllabus for my sections differs slightly
from those used in other sections

\hypertarget{changelog}{%
\section{Changelog}\label{changelog}}

This syllabus and schedule are subject to change in the event of
extenuating circumstances. If you are absent from class, it is your
responsibility to check on announcements made while you were absent.
Changes and corrections are listed in the changelog below and will be
announced on Canvas.

\begin{itemize}
\tightlist
\item
  2020-06-22: Draft created
\end{itemize}

\hypertarget{general-information}{%
\section{General information}\label{general-information}}

\begin{itemize}
\tightlist
\item
  \textbf{Course name and number}: CHEM 129A (2 Units)
\item
  \textbf{Prerequisites}: CHEM 8 or CHEM 128A with a grade of C or
  better. CHEM 128A can be taken concurrently.
\item
  \textbf{Contact}\footnote{Please note that I typically check email
    between 11 am and 4 pm, Monday through Friday. Usually, my response
    time is \emph{within 12 hours of reading the message}. I strongly
    encourage you to use Canvas communication tools (Discussions) where
    you can ask questions to the entire class at any time, making it
    more likely to get a quick response.}: Science 1 room 352, phone
  (559) 278-2711, \url{hmuchalski@mail.fresnostate.edu}
\item
  \textbf{Office Hours}: TBA. Additional consultation appointments can
  be scheduled through calendar function ``Find Appointments'' on
  Canvas.
\end{itemize}

\hypertarget{course-materials-and-technology}{%
\section{Course materials and
technology}\label{course-materials-and-technology}}

\begin{itemize}
\tightlist
\item
  \textbf{Textbook:} ``A Micro-scale Approach to Organic Laboratory
  Techniques, 6th edition'' by Donald Pavia et al.~published by
  Thompson/Brooks Cole. Previous editions will also be sufficient to
  learn the material but page numbers as well as problems will be
  different.
\item
  \textbf{Notebook:} Organic chemistry laboratory notebook from
  Hayden-McNeil, spiral-bound, (ISBN:9781930882461). General chemistry
  notebook (ISBN:978-1930882744) is also good alternative but will not
  have organic chemistry-specific reference materials on covers.
\item
  \textbf{Personal protective equipment (PPE)}: Lab coat and approved
  safety goggles. Disposable nitrile gloves will be provided.
\item
  \textbf{Canvas:} The central repository for all course materials and
  information is our Canvas site, accessible through
  \url{https://fresnostate.instructure.com/courses/3782}. The Canvas
  site will house your grades, links to handouts, videos, and other
  materials.
\end{itemize}

\hypertarget{laboratory-code-of-safe-practices}{%
\section{Laboratory Code of Safe
Practices}\label{laboratory-code-of-safe-practices}}

\begin{enumerate}
\def\labelenumi{\arabic{enumi}.}
\tightlist
\item
  NO food or drink in the laboratory.
\item
  Wear clothing appropriate for laboratory work.
\item
  Select and correctly use appropriate Personal Protective Equipment
  (PPE).
\item
  Know what to do and who to contact in an emergency in the laboratory.
\item
  Avoid distractions and be alert to and aware of your surroundings and
  potential hazards in your area.
\item
  Maintain a safe and clean work area.
\item
  Only conduct experiments or procedures approved by your lab instructor
  or research advisor.
\item
  Understand the common chemical hazards and hazards specific to the
  chemicals and procedures with which you are working.\\
\item
  Understand and follow best practices on how to handle, transport,
  store, and dispose of chemicals safely.
\item
  If any equipment, glassware, or procedures are not working properly or
  as expected, notify your instructor before proceeding.
\item
  Notify your instructor if you have, develop, or may develop any
  medical conditions (e.g.~severe asthma, limited mobility, vision
  impairment, pregnancy, etc) that may affect your safety in the
  laboratory or sensitivity to chemicals, so that your instructor can
  properly advise or accommodate you on minimizing the risks associated
  with laboratory work.
\end{enumerate}

Full discussion of these principles can be found here:
\url{https://goo.gl/1UFRbo}. For additional information about safety in
undergraduate teaching labs please refer to
\href{https://www.acs.org/content/dam/acsorg/about/governance/committees/chemicalsafety/publications/acs-safety-guidelines-academic.pdf}{\emph{Guidelines
for Chemical Laboratory Safety in Academic Institutions}} published by
American Chemical Society.

\hypertarget{what-is-chem-129a}{%
\section{What is CHEM 129A}\label{what-is-chem-129a}}

CHEM 129A is a two unit introductory laboratory course in organic
chemistry. It is primarily concerned with introducing the tools and
techniques that chemists use to synthesize and investigate the
properties of organic compounds (see the \protect\hyperlink{slo}{list
below}). Some of these techniques are the same or similar to those you
learned in general chemistry courses but may be modified because the
experiments use very small amounts of material (micro-scale techniques).

Students who successfully complete CHEM 129A generally enroll in CHEM
129B, which further develops students' laboratory skills. Some students
then continue with CHEM 190, undergraduate research or independent
study.

\hypertarget{slo}{%
\subsection{Student Learning Outcomes}\label{slo}}

Students who successfully complete CHEM 129A should be able to:

\begin{itemize}
\tightlist
\item
  maintain an accurate laboratory notebook that would allow another
  properly trained person to reproduce the experimental work with the
  similar results;
\item
  understand how to work safely in the laboratory, including the
  disposal of chemical wastes;
\item
  find relevant information about reagents, equipment, and techniques;
\item
  build apparatus for a reflux reaction;
\item
  carry out basic organic techniques such as extraction,
  crystallization, distillation, and chromatography;
\item
  measure physical properties of organic compounds including melting and
  boiling points;
\item
  carry out liquid--liquid extraction; and
\item
  analyze purity and chemical identity of organic compounds using TLC,
  GC, and IR.
\end{itemize}

\hypertarget{course-requirements}{%
\section{Course requirements}\label{course-requirements}}

\hypertarget{in-lab-requirements}{%
\subsection{In-lab requirements}\label{in-lab-requirements}}

Two 170-minute lab meetings per week (Monday and Wednesday) where we
experimentally explore selected organic chemistry concepts (5.67 hours
per week). Attendance is mandatory because you must complete all
experiments to obtain a passing grade. Typically, the in-lab session
will start with a short review of relevant material, concepts, and
chemical safety through group discussions and problem-solving
activities.

You will not be allowed to enter the lab if a) you are late and missed
the pre-lab discussion; b) you are not appropriately dressed to work in
the lab; or 3) you don't complete the pre-lab assignment.

If you cannot attend a lab you must notify your instructor and lab
coordinator as soon as possible explaining why you have to miss the lab.
Arranging make-ups is very difficult because every semester all lab
sections are at full capacity. If your request is approved you are still
required to document the reason for missing the lab, and will be
expected to make the lab up and complete the assigned work. If the above
conditions are not met, you will receive 0 for the missed lab.

\hypertarget{out-of-class-requirements}{%
\subsection{Out-of-class requirements}\label{out-of-class-requirements}}

I expect that you will come to lab with a basic understanding of the
concepts that will allow you to fully participate in pre-lab
discussions. Preparing for the lab means making honest effort to learn
on your own. This is done by:

\begin{itemize}
\tightlist
\item
  reading the assigned textbook material related to the scheduled
  experiment;
\item
  preparing the pre-lab assignment;
\item
  reviewing the relevant techniques for each experiment (listed in the
  introduction section for each experiment);
\item
  watching videos demonstrating laboratory techniques; and
\item
  reviewing safety data sheets (SDS) for chemicals you will be working
  with.
\end{itemize}

Out-of-class work also includes analysis of data collected in the lab.
For the first half of the semester this will take form of post-lab
summary written in your notebook. Later in the semester the post-lab
analysis will include a typewritten lab report. Total out-of-class
workload is estimated at 8--12 hours per week.

\hypertarget{types-of-graded-work}{%
\section{Types of graded work}\label{types-of-graded-work}}

There are four kinds of graded work you will encounter in this course:

\begin{enumerate}
\def\labelenumi{\arabic{enumi}.}
\tightlist
\item
  Laboratory notes (both pre-lab and in-lab notes)
\item
  Post-lab summaries and lab reports
\item
  Quizzes
\item
  Practicals
\end{enumerate}

\hypertarget{laboratory-notebook}{%
\subsection{Laboratory notebook}\label{laboratory-notebook}}

Assume that you're in note-taking mode at all time when you are in the
lab. Well prepared pre-lab and in-lab notes are very important because
they are the basis for the post-lab summary or report that you will
write for each set of experiments. Pre-lab notes are generated in
preparation for the scheduled experiment. In-lab notes document what you
did in the lab (including notes taken during lecture and discussions) as
well as results of your experimental work. To know what to write and not
to write in the notebook is a balancing act of relevance and brevity.
You need only details that are relevant and necessary to reproduce the
experiment. You can assume that your notes are going to be read by a
trained organic chemist. For example, you don't have to explain what is
a round-bottomed flask. Keep all prints (IR and NMR spectra, gas
chromatograms, etc.) and sketches of TLC plates as part of you notes
portfolio. Your lab notebook is subject to evaluation at any time and
must be kept current. Instructions on how to keep laboratory notebook
are discussed in the textbook and will be reviewed in the lab. You will
submit copies of your pre-lab notes at the beginning of the lab and
in-lab notes before you leave the lab room.

\hypertarget{post-lab-summaries-and-reports}{%
\subsection{Post-lab summaries and
reports}\label{post-lab-summaries-and-reports}}

One of the main objectives of this course is for students to learn to
interpret and communicate the meaning of experimental results. The
post-lab summary or lab report is an assignment that shows the depth of
your understanding of the concepts, techniques, and instrumentation used
in the lab. For experiments in the first half of the semester you will
write a short summary of results and conclusions in your notebook. All
notebook pages will be graded as a whole (5\% per lab).

\hypertarget{lab-reports}{%
\subsection{Lab reports}\label{lab-reports}}

Later in the semester, a typewritten report will be assigned as post-lab
summary and reflection. There will be total of six (6) typewritten
reports. Four regular experiments and two practicals. I will use a
grading system that allows for revision and resubmission that gives you
multiple attempts to demonstrate the level of learning you achieved. Lab
reports will be evaluated using a rubric that classifies the work with
marks of \emph{Satisfactory} (S) \emph{Progressing} (P), or
\emph{Incomplete} (I). Work marked as \emph{Satisfactory} will receive
full credit. Reports marked as \emph{Incomplete} or \emph{Progressing}
can be revised and resubmitted before the grade becomes final. Each
student can re-submit up to two reports (or one report twice) without
point reduction. Additional revisions will be at a cost of point
reduction. Details of the process of revision and resubmission will be
discussed in class.

\hypertarget{quizzes}{%
\subsection{Quizzes}\label{quizzes}}

Quizzes test your understanding of the lab techniques and experiments
you conducted and will be based on material covered in the pre-lab
lectures, reading assignments, and experiments. Quizzes will be
take-home assignments because they test higher order thinking skills
which are rarely well performed under the pressure of time. The only
quiz we will have in class is the \emph{Safety Quiz}.

\textbf{Safety Quiz}. You must pass the Safety Quiz to remain in the
class. You can take the quiz up to 3 times. The quiz is based upon the
information on safety sheet, your instructor's lab lecture on safety,
and Technique 1 (Safety) in the lab textbook. You must be familiar with
the safety protocols, abide by the guidelines \emph{at all time} to keep
yourself, other lab occupants, and everyone else in the building safe.

\hypertarget{practicals}{%
\subsection{Practicals}\label{practicals}}

These are the last two experiments for the semester (see
\href{https://hmlab.page.link/129a-schedule}{Lab Schedule}). You will be
given a procedure (handout) at the beginning of the Practical session
and your grade for these Practicals will be based on your skills and
performance in the lab (yield and purity of your product) and your
experimental write-up.

\hypertarget{final-letter-grade-scheme}{%
\section{Final letter grade scheme}\label{final-letter-grade-scheme}}

Grade brackets are imposed by course coordinator. In the past, the
grading scale followed a pattern close to the following: A = 85--100, B
75--84, C 65--74; D 50--64; and F \textless50.

\begin{longtable}[]{@{}lll@{}}
\toprule
Grade component & \% (each) & Subtotal\tabularnewline
\midrule
\endhead
Experiments (10) & 5\% & 50\%\tabularnewline
Practicals (2) & 9\% & 18\%\tabularnewline
Quizzes (4) & 8\% & 32\%\tabularnewline
\bottomrule
\end{longtable}

\hypertarget{course-policies}{%
\section{Course policies}\label{course-policies}}

\hypertarget{technology-issues-when-submitting-work}{%
\subsection{Technology issues when submitting
work}\label{technology-issues-when-submitting-work}}

For assignments submitted electronically, it is your responsibility to
make sure they are submitted on time, through any means necessary, even
if technology issues arise. If a tech issue arises, it is your
responsibility to find another way to get it to me (for example, via an
email attachment). Technology issues that are avoidable or resolved with
a simple work-around will not be considered valid grounds for a deadline
extension. For example, if you are trying to upload a Lab to Canvas and
Canvas won't accept the file, you should try again later or send the
file as an email attachment until you can upload it successfully.

\hypertarget{academic-dishonesty}{%
\subsection{Academic Dishonesty}\label{academic-dishonesty}}

For most assignments you are allowed and encouraged to work with others.
However, the final product that you submit for feedback must be the
result of your own efforts. Therefore you may share ideas and strategies
with others, but collaboration on the actual finished product you submit
is not allowed. Your work is expected to be the product of your own
thinking, written and explained in your own words with no parts of the
work copied from external sources such as books or websites, and done
clearly enough in your own mind that you could explain the work from
start to finish if asked. Specifically, this excludes:

\begin{itemize}
\tightlist
\item
  copying work from another student;
\item
  copying work from a website;
\item
  paraphrasing work done by another student or from print or internet
  resources---i.e.~putting it in your own words---without coming up with
  the main ideas and strategies yourself; and
\item
  \emph{allowing or enabling} another student to copy or paraphrase work
  that you did, even if you did the original work yourself.
\end{itemize}

Violation of this policy is considered ``academic dishonesty'' and
carries with it strong punitive measures mandated by Fresno State,
including possible automatic failure of the course or suspension from
the university. For details, please see APM 235 by going to
\url{http://www.fresnostate.edu/aps/documents/apm/235.pdf}.

You may feel tempted to academic dishonesty at some point in the
semester. The work can be difficult, and many of you are under a lot of
stress. If you are considering academic dishonesty, please STOP, take a
breath, and remember that your classmates and I want you to succeed in
the course. You are not alone, and you have a strong network in the
class for getting help. The revision and resubmission policies mean that
it's OK to turn in work that isn't perfect. There is no need to be
academically dishonest! Just do your best on the work, and you'll have
the chance to revise it later.

\hypertarget{dropping-the-course-after-the-census-date}{%
\subsection{Dropping the course after the census
date}\label{dropping-the-course-after-the-census-date}}

A \emph{serious and compelling reason} is defined as an unexpected
condition that is not present prior to enrollment in the course that
unexpectedly arises and interferes with a student's ability to attend
class meetings and/or complete course requirements. The reason must be
acceptable to and verified by the instructor of record and the
department chair. The condition must be stated in writing on the
appropriate form. The student must provide documentation that
substantiates the condition.

Failing or performing poorly in a class is not an acceptable ``serious
and compelling reason'' within the University policy, nor is
dissatisfaction with the subject matter, class or instructor.

\hypertarget{university-policies-and-disclaimers}{%
\section{University policies and
disclaimers}\label{university-policies-and-disclaimers}}

In addition to course policies, you are expected to be familiar with
Academic Regulations described in the
\href{http://www.fresnostate.edu/catalog/academic-regulations/}{University
Catalog} as well as policies listed below.

\textbf{Students with Disabilities}: Upon identifying themselves to the
instructor and the university, students with disabilities will receive
reasonable accommodation for learning and evaluation. For more
information, contact Services to Students with Disabilities in the Henry
Madden Library, Room 1202 (278-2811).

\begin{itemize}
\tightlist
\item
  Class Schedule Policies:
  \url{http://fresnostate.edu/studentaffairs/classschedule/policy/}
\item
  Copyright Policy: \url{http://libguides.csufresno.edu/copyright}
\item
  Students with Disabilities:
  \url{http://fresnostate.edu/studentaffairs/careers/students/interests/disabilities.html}
\item
  Academic Integrity and Honor Code:
  \url{http://www.fresnostate.edu/academics/facultyaffairs/documents/apm/236.pdf}
\item
  Policy on Cheating and Plagiarism:
  \url{http://fresnostate.edu/studentaffairs/studentconduct/policies/cheating-plagiarism.html}
\item
  Add/Drop Course:
  \url{http://www.fresnostate.edu/studentaffairs/registrar/registration/}
\item
  Computer requirements:
  \url{https://www.fresnostate.edu/catalog/academic-regulations/index.html\#computerreq}
\item
  Disruptive classroom behavior:
  \url{http://www.fresnostate.edu/academics/facultyaffairs/documents/apm/419.pdf}
\end{itemize}

\hypertarget{university-services}{%
\section{University Services}\label{university-services}}

\begin{itemize}
\tightlist
\item
  \href{http://fresnostateasi.org/}{Associated Students, Inc.}
\item
  \href{http://fresnostate.edu/studentaffairs/dsc/index.html}{Dream
  Success Center}
\item
  \href{http://fresnostate.edu/studentaffairs/lrc}{Learning Center
  Information}
\item
  \href{https://www.fresnostate.edu/studentaffairs/health/}{Student
  Health and Counseling Center}
\item
  \href{http://www.fresnostate.edu/artshum/writingcenter/}{Writing
  Center}
\end{itemize}
