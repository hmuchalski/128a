\hypertarget{general-information}{%
\section{General information}\label{general-information}}

\begin{itemize}
\tightlist
\item
  \textbf{Course name and number}: Strategies and Tactics in Organic
  Synthesis; CHEM 240T (graduate, 2 Units)
\item
  \textbf{Contact}: Science 1 room 352, phone (559) 278-2711,
  \url{hmuchalski@mail.fresnostate.edu}
\item
  \textbf{Canvas:} The central repository for all course materials and
  information is our Canvas site, accessible through
  \url{https://fresnostate.instructure.com/courses/4365}. The Canvas
  site will house your grades, links to handouts, videos, and other
  materials.
\item
  \textbf{Textbook:} There is no specific textbook for this class but I
  recommend you get a copy of organic chemistry textbook to use as
  general reference. It doesn't have to be recent. Older edition of any
  major textbook will work just as well.
\item
  \textbf{Tech:} Scifinder scholar account to access and search the
  database. EndNote, and ChemDraw, both available for free for Fresno
  State students. Refer to instructions on how to obtain the software
  (Canvas).
\item
  \textbf{Office Hours:} I will be available for consultations after
  each class meeting. Walk-in office hours are Monday and Wednesday
  12:00--01:00 pm. Additional consultation appointments can be scheduled
  through calendar function ``Find Appointments'' on Canvas.
\end{itemize}

\hypertarget{introduction}{%
\section{Introduction}\label{introduction}}

When you took the sophomore course in organic chemistry, you studied the
typical reactions of halooalkanes, alkenes/alkynes, aromatic compounds
and organic carbonyl compounds, etc. That was to introduce you to
compound classes and to give you an understanding of how and why such
reactions occur. However, the course did not focus how to design a
synthetic route to prepare a given organic compound from readily
available starting materials. This is a crucial aspect of organic
chemistry. The design and synthesis of novel organic compounds is
fundamental to the development of new medicines, agrochemicals,
plastics, dyestuffs, etc. This course will introduce you to the ideas
involved in synthetic design, particularly `retrosynthetic analysis'
(RSA).

\hypertarget{slo}{%
\subsection{Student Learning Outcomes}\label{slo}}

Students who successfully complete this course should be able to:

\begin{itemize}
\tightlist
\item
  be familiar with the terminology of retrosynthetic analysis (RSA);
\item
  distinguish factors such as chemoselectivity, regioselectivity,
  stereoselectivity and protecting group methodology and their
  importance in synthetic design;
\item
  identify synthons and their synthetic equivalents and functional group
  interconversions;
\item
  use RSA to design and evaluate syntheses of target molecules of
  medium~complexity
\item
  evaluate a synthetic plan and identify flaws in synthetic design
\item
  propose a plausible reaction mechanism for a given reactions
\end{itemize}

\hypertarget{topics}{%
\subsection{Topics}\label{topics}}

\begin{itemize}
\tightlist
\item
  Retrosynthetic Analysis (RSA)
\item
  Disconnection via Functional Group Interconversion
\item
  Protecting Groups
\item
  Carbon--Carbon Bond Disconnections
\item
  Synthesis of Compounds Containing Rings
\item
  Stereochemistry: Prediction and Control
\item
  Cross-Coupling Reactions
\item
  Two-Direactional Synthesis
\item
  Catalysis
\end{itemize}

\hypertarget{what-to-expect}{%
\section{What to expect}\label{what-to-expect}}

The learning mode that will dominate our class meetings is deep analysis
of published syntheses of case studies. The goal of this course is not
to cover all strategies and tactics used in organic synthesis but to
provide opportunities for development of a skill which you need to
master as a graduate student: to quickly learn complex and unfamiliar
science topics to the degree that you can teach them to others.

Active participation in class discussions is key to getting the most out
of this class. Expect to be called to the board on a weekly basis to
discuss synthetic and mechanistic problems. Don't be concerned if you
feel weak in some areas. By working through a problem you will
strengthen your understanding and refine thinking process. This means
that it is of utter importance that you come to class prepared.

Reading, researching, and working through problems should be your
primary out-of-class activities.

\hypertarget{types-of-graded-work}{%
\section{Types of graded work}\label{types-of-graded-work}}

There are three types of graded work you will encounter in this course:

\begin{enumerate}
\def\labelenumi{\arabic{enumi}.}
\tightlist
\item
  Problem sets
\item
  Midterm take-home exam
\item
  Synthesis proposal
\item
  Proposal presentation
\end{enumerate}

Problem sets are written homework assignments containing synthetic
problems and mechanisms. We will review some of those problems and
students will be asked to go to the board and solve the problem in front
of the group. Thus, just having the answer may not be sufficient to
receive full credit on the assignment if you are not able to develop the
solution again.

The take-home exam will be just like problem sets but bigger (more
problems) and more comprehensive.

Each student will also propose total synthesis of natural product and
present it to the class.

\hypertarget{final-letter-grade-scheme}{%
\section{Final letter grade scheme}\label{final-letter-grade-scheme}}

Grade brackets are imposed by course coordinator. In the past, the
grading scale followed a pattern close to the following: A = 90--100, B
80--89, C 70--79; D 60--69; and F \textless60.

\begin{longtable}[]{@{}ll@{}}
\toprule
Grade component & Subtotal\tabularnewline
\midrule
\endhead
Problem Sets & 50\%\tabularnewline
Midterm take-home exam & 20\%\tabularnewline
Synthesis Proposal & 15\%\tabularnewline
Presentation & 15\%\tabularnewline
\bottomrule
\end{longtable}

\hypertarget{course-policies}{%
\section{Course policies}\label{course-policies}}

\hypertarget{technology-issues-when-submitting-work}{%
\subsection{Technology issues when submitting
work}\label{technology-issues-when-submitting-work}}

For assignments submitted electronically, it is your responsibility to
make sure they are submitted on time, through any means necessary, even
if technology issues arise. If a tech issue arises, it is your
responsibility to find another way to get it to me (for example, via an
email attachment). Technology issues that are avoidable or resolved with
a simple work-around will not be considered valid grounds for a deadline
extension.

\hypertarget{academic-dishonesty}{%
\subsection{Academic Dishonesty}\label{academic-dishonesty}}

For most assignments you are allowed and encouraged to work with others.
However, the final product that you submit for feedback must be the
result of your own efforts. Therefore you may share ideas and strategies
with others, but collaboration on the actual finished product you submit
is not allowed. Your work is expected to be the product of your own
thinking, written and explained in your own words with no parts of the
work copied from external sources such as books or websites, and done
clearly enough in your own mind that you could explain the work from
start to finish if asked. Specifically, this excludes:

\begin{itemize}
\tightlist
\item
  copying work from another student;
\item
  copying work from a website;
\item
  paraphrasing work done by another student or from print or internet
  resources---i.e.~putting it in your own words---without coming up with
  the main ideas and strategies yourself; and
\item
  \emph{allowing or enabling} another student to copy or paraphrase work
  that you did, even if you did the original work yourself.
\end{itemize}

Violation of this policy is considered ``academic dishonesty'' and
carries with it strong punitive measures mandated by Fresno State,
including possible automatic failure of the course or suspension from
the university. For details, please see APM 235 by going to
\url{http://www.fresnostate.edu/aps/documents/apm/235.pdf}.

You may feel tempted to academic dishonesty at some point in the
semester. The work can be difficult, and many of you are under a lot of
stress. If you are considering academic dishonesty, please STOP, take a
breath, and remember that your classmates and I want you to succeed in
the course. You are not alone, and you have a strong network in the
class for getting help.

\hypertarget{dropping-the-course-after-the-census-date}{%
\subsection{Dropping the course after the census
date}\label{dropping-the-course-after-the-census-date}}

A \emph{serious and compelling reason} is defined as an unexpected
condition that is not present prior to enrollment in the course that
unexpectedly arises and interferes with a student's ability to attend
class meetings and/or complete course requirements. The reason must be
acceptable to and verified by the instructor of record and the
department chair. The condition must be stated in writing on the
appropriate form. The student must provide documentation that
substantiates the condition.

Failing or performing poorly in a class is not an acceptable ``serious
and compelling reason'' within the University policy, nor is
dissatisfaction with the subject matter, class or instructor.

\hypertarget{university-policies-and-disclaimers}{%
\section{University policies and
disclaimers}\label{university-policies-and-disclaimers}}

In addition to course policies, you are expected to be familiar with
Academic Regulations described in the
\href{http://www.fresnostate.edu/catalog/academic-regulations/}{University
Catalog} as well as policies listed below.

\textbf{Students with Disabilities}: Upon identifying themselves to the
instructor and the university, students with disabilities will receive
reasonable accommodation for learning and evaluation. For more
information, contact Services to Students with Disabilities in the Henry
Madden Library, Room 1202 (278-2811).

\begin{itemize}
\tightlist
\item
  Class Schedule Policies:
  \url{http://fresnostate.edu/studentaffairs/classschedule/policy/}
\item
  Copyright Policy: \url{http://libguides.csufresno.edu/copyright}
\item
  Students with Disabilities:
  \url{http://fresnostate.edu/studentaffairs/careers/students/interests/disabilities.html}
\item
  Academic Integrity and Honor Code:
  \url{http://www.fresnostate.edu/academics/facultyaffairs/documents/apm/236.pdf}
\item
  Policy on Cheating and Plagiarism:
  \url{http://fresnostate.edu/studentaffairs/studentconduct/policies/cheating-plagiarism.html}
\item
  Add/Drop Course:
  \url{http://www.fresnostate.edu/studentaffairs/registrar/registration/}
\item
  Computer requirements:
  \url{https://www.fresnostate.edu/catalog/academic-regulations/index.html\#computerreq}
\item
  Disruptive classroom behavior:
  \url{http://www.fresnostate.edu/academics/facultyaffairs/documents/apm/419.pdf}
\end{itemize}

\hypertarget{university-services}{%
\section{University Services}\label{university-services}}

\begin{itemize}
\tightlist
\item
  \href{http://fresnostateasi.org/}{Associated Students, Inc.}
\item
  \href{http://fresnostate.edu/studentaffairs/dsc/index.html}{Dream
  Success Center}
\item
  \href{http://fresnostate.edu/studentaffairs/lrc}{Learning Center
  Information}
\item
  \href{https://www.fresnostate.edu/studentaffairs/health/}{Student
  Health and Counseling Center}
\item
  \href{http://www.fresnostate.edu/artshum/writingcenter/}{Writing
  Center}
\end{itemize}
