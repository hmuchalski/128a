\newpage

Welcome to CHEM 129B, Organic Chemistry Laboratory 2! This document
contains all the information you need to know about the course.
\textbf{Read this document carefully, familiarize yourself with how the
course works, and maintain that familiarity throughout the term.} It is
nearly 1,700 words for a reason. Almost all questions about the course
that you might ask can be answered by referencing the syllabus.

\hypertarget{changelog}{%
\section{Changelog}\label{changelog}}

This syllabus and schedule are subject to change in the event of
extenuating circumstances. If you are absent from class, it is your
responsibility to check on announcements made while you were absent.
Changes and corrections are listed in the changelog below and will be
announced on Canvas.

\begin{itemize}
\tightlist
\item
  2020-06-03: Syllabus updated and published on Canvas
\end{itemize}

\hypertarget{general-information}{%
\section{General information}\label{general-information}}

\begin{itemize}
\tightlist
\item
  \textbf{Course name and number}: CHEM 129B (2 Units)
\item
  \textbf{Prerequisites}: CHEM 128B with a grade of C or better (can be
  taken concurrently) 
\item
  \textbf{Phone}: (559) 278-2711\footnote{This is my campus office
    number which redirects to my personal cell phone. Calling me is the
    most direct way to reach me. Sometimes 3-minute phone call can solve
    a problem that multiple email exchanges cannot.}
\item
  \textbf{Email}: \url{hmuchalski@mail.fresnostate.edu}\footnote{Please
    note that I typically check email between 11 am and 6 pm, Monday
    through Friday. Usually, my response time is \emph{within 12 hours
    of reading the message}. We also have online course tools where you
    can ask questions to the entire class at any time, making it more
    likely to get a quick response.}
\item
  \textbf{Office Hours}: I will be available for consultations for 30
  min after each class meeting. Walk-in office hours are Monday and
  Wednesday 12:00--01:00 pm. Additional consultation appointments can be
  scheduled through calendar function ``Find Appointments'' on Canvas.
\end{itemize}

\begin{itemize}
\tightlist
\item
  \textbf{Techniques Reference Manual} There are several options here.
  See Canvas for the full list of resources 
\item
  \textbf{Canvas:} The central repository for all course materials and
  information is our Canvas site, accessible through
  \url{https://fresnostate.instructure.com/courses/24394}. The Canvas
  site will house your grades, links to handouts, videos, and other
  materials.
\item
  \textbf{Personal computer:} You will need a x86 class personal
  computer, either Windows of macOS that can run desktop applications.
  Mobile devices have potential to augment the learning experience, but
  are not capable to run the apps we will use. Refer to Canvas for the
  computer and software requirements.
\end{itemize}

\hypertarget{course-learning-outcomes}{%
\subsection{Course Learning Outcomes}\label{course-learning-outcomes}}

This course is organized into Modules and aims to help students achieve
learning outcomes through online activities. Each module contains
materials you need to complete the associated activities. Each
course-level student learning outcome is aligned with department-level
learning outcomes which are modeled after the standards set by the
American Chemical Society.

Upon completion of this course students will be able to:

\begin{enumerate}
\def\labelenumi{\arabic{enumi}.}
\tightlist
\item
  Communicate the structure and properties of organic molecules using
  common drawing and naming conventions
\item
  Analyze chemical structures and reaction conditions to make and defend
  predictions about chemical transformations
\item
  Use online databases to find relevant research articles containing
  information such as physicochemical properties of organic molecules,
  synthetic procedures, and spectroscopic data.
\item
  Use software tools such as ChemDraw to draw chemical structures,
  reactions, and mechanisms
\item
  Use software tools such as MestReNova and Topspin to process raw NMR
  data
\item
  Plan a synthesis experiment by evaluating the information found in
  online databases and research articles 
\item
  Analyze the results of an experiment and be able to identify sources
  of error and suggest improvements;
\item
  Interpret spectroscopic data of organic compounds to confirm the
  structure of organic compounds
\item
  Communicate the results of experiments to the instructor and peers in
  a written form (lab report)
\item
  Communicate the results of experiments to the instructor and peers in
  an oral or poster presentation
\end{enumerate}

\hypertarget{department-student-learning-outcomes}{%
\subsection{Department Student Learning
Outcomes}\label{department-student-learning-outcomes}}

\begin{enumerate}
\def\labelenumi{\arabic{enumi}.}
\tightlist
\item
  Students will apply their understanding of terminology, concepts,
  theories, and skills to solve problems by defining problems and
  research questions clearly, formulating testable hypotheses, designing
  and conducting experimental tests of hypotheses, analyzing and
  interpreting data, and drawing appropriate conclusions within
  professional ethical guidelines. (ACS Standards 7.1 \& 7.6)
\item
  Students will demonstrate the ability to conduct laboratory work of
  high quality including handling chemicals and other laboratory hazards
  in a safe, ethical, and socially responsible manner, keeping accurate,
  clear, concise, and complete records of their laboratory work in a
  notebook, properly using standard laboratory equipment and
  instruments, and evaluating the reliability and significance of
  laboratory data, all within professional ethical guidelines. (ACS
  Standards 7.1, 7.3, 7.6)
\item
  Students will complete a literature search in one or more of the five
  chemical sub disciplines by using common literature search techniques
  and tools to find recent journal articles from the peer-reviewed
  literature, critically read these articles to extract relevant
  information, and communicate the significance of these articles in
  written or oral formats within professional ethical guidelines. (ACS
  Standards 7.2 \& 7.6)
\item
  Students will demonstrate the ability to clearly and effectively
  communicate their scientific results and opinions using written
  formats while following professional style and format conventions
  within professional ethical guidelines. (ACS Standards 7.4 \& 7.6)
\item
  Students will demonstrate the ability to clearly and effectively
  communicate their scientific results and opinions using oral formats
  while following professional style and format conventions within
  professional ethical guidelines. (ACS Standards 7.4 \& 7.6)
\item
  Students will demonstrate the ability to function effectively in
  collaborative and group work environments including the ability to
  work on a component of a larger project and connect work with previous
  results within professional ethical standards. (ACS Standard 7.5 \&
  7.6)
\end{enumerate}

\hypertarget{course-modules}{%
\section{Course Modules}\label{course-modules}}

\begin{enumerate}
\def\labelenumi{\arabic{enumi}.}
\tightlist
\item
  Chemical Hazards and Risk Assessment
\item
  Journals and Scientific Databases
\item
  SciFinder Scholar
\item
  Managing References
\item
  Green Chemistry
\item
  Spectroscopy
\item
  Synthesis of Methyl Diantilis and its Derivatives
\item
  Greener Synthesis of Benzil\\
\item
  Stereochemistry of Reduction of Benzil with Sodium Borohydride
\item
  Structural Elucidation of Acetophenone Derivatives
\item
  Independent Project
\end{enumerate}

\hypertarget{assignments-and-grading}{%
\section{Assignments and grading}\label{assignments-and-grading}}

Table below lists categories of assignment and weights associated with
each group.

\begin{longtable}[]{@{}ll@{}}
\toprule
Item & Value\tabularnewline
\midrule
\endhead
Lab Safety & 12\%\tabularnewline
Research Tools & 8\%\tabularnewline
Quizzes \& Homework & 15\%\tabularnewline
Spectroscopy & 15\%\tabularnewline
Experiments & 36\%\tabularnewline
Independent Project & 14\%\tabularnewline
TOTAL & \textbf{100\%}\tabularnewline
\bottomrule
\end{longtable}

Grade brackets are imposed by course coordinator. In the past, the
grading scale followed a pattern close to the following: A = 90--100, B
80--89, C 70--79; D 60--69; and F \textless60.

\hypertarget{course-policies}{%
\section{Course policies}\label{course-policies}}

\hypertarget{technology-issues-when-submitting-work}{%
\subsection{Technology issues when submitting
work}\label{technology-issues-when-submitting-work}}

For assignments submitted electronically, it is your responsibility to
make sure they are submitted on time, through any means necessary, even
if technology issues arise. If a tech issue arises, it is your
responsibility to find another way to get it to me (for example, via an
email attachment). Technology issues that are avoidable or resolved with
a simple work-around will not be considered valid grounds for a deadline
extension. For example, if you are trying to upload a Lab to Canvas and
Canvas won't accept the file, you should try again later, use a
different browser, or send the file as an email attachment until you can
upload it successfully.

\hypertarget{academic-dishonesty}{%
\subsection{Academic Dishonesty}\label{academic-dishonesty}}

For most assignments you are allowed and encouraged to work with others.
However, the final product that you submit for feedback must be the
result of your own efforts. Therefore you may share ideas and strategies
with others, but collaboration on the actual finished product you submit
is not allowed. Your work is expected to be the product of your own
thinking, written and explained in your own words with no parts of the
work copied from external sources such as books or websites, and done
clearly enough in your own mind that you could explain the work from
start to finish if asked. Specifically, this excludes:

\begin{itemize}
\tightlist
\item
  copying work from another student;
\item
  copying work from a website;
\item
  paraphrasing work done by another student or from print or internet
  resources---i.e.~putting it in your own words---without coming up with
  the main ideas and strategies yourself; and
\item
  \emph{allowing or enabling} another student to copy or paraphrase work
  that you did, even if you did the original work yourself.
\end{itemize}

Violation of this policy is considered ``academic dishonesty'' and
carries with it strong punitive measures mandated by Fresno State,
including possible automatic failure of the course or suspension from
the university. For details, please see APM 235 by going to
\url{http://www.fresnostate.edu/aps/documents/apm/235.pdf}.

You may feel tempted to academic dishonesty at some point in the
semester. The work can be difficult, and many of you are under a lot of
stress. If you are considering academic dishonesty, please STOP, take a
breath, and remember that your classmates and I want you to succeed in
the course. You are not alone, and you have a strong network in the
class for getting help. The revision and resubmission policies mean that
it's OK to turn in work that isn't perfect. There is no need to be
academically dishonest! Just do your best on the work, and you'll have
the chance to revise it later.

\hypertarget{dropping-the-course-after-the-census-date}{%
\subsection{Dropping the course after the census
date}\label{dropping-the-course-after-the-census-date}}

A \emph{serious and compelling reason} is defined as an unexpected
condition that is not present prior to enrollment in the course that
unexpectedly arises and interferes with a student's ability to attend
class meetings and/or complete course requirements. The reason must be
acceptable to and verified by the instructor of record and the
department chair. The condition must be stated in writing on the
appropriate form. The student must provide documentation that
substantiates the condition.

Failing or performing poorly in a class is not an acceptable ``serious
and compelling reason'' within the University policy, nor is
dissatisfaction with the subject matter, class or instructor.

\hypertarget{university-policies-and-disclaimers}{%
\section{University policies and
disclaimers}\label{university-policies-and-disclaimers}}

In addition to course policies, you are expected to be familiar with
Academic Regulations described in the
\href{http://www.fresnostate.edu/catalog/academic-regulations/}{University
Catalog} as well as policies listed below.

\textbf{Students with Disabilities}: Upon identifying themselves to the
instructor and the university, students with disabilities will receive
reasonable accommodation for learning and evaluation. For more
information, contact Services to Students with Disabilities in the Henry
Madden Library, Room 1202 (278-2811).

\begin{itemize}
\tightlist
\item
  Class Schedule Policies:
  \url{http://fresnostate.edu/studentaffairs/classschedule/policy/}
\item
  Copyright Policy: \url{http://libguides.csufresno.edu/copyright}
\item
  Students with Disabilities:
  \url{http://fresnostate.edu/studentaffairs/careers/students/interests/disabilities.html}
\item
  Academic Integrity and Honor Code:
  \url{http://www.fresnostate.edu/academics/facultyaffairs/documents/apm/236.pdf}
\item
  Policy on Cheating and Plagiarism:
  \url{http://fresnostate.edu/studentaffairs/studentconduct/policies/cheating-plagiarism.html}
\item
  Add/Drop Course:
  \url{http://www.fresnostate.edu/studentaffairs/registrar/registration/}
\item
  Computer requirements:
  \url{https://www.fresnostate.edu/catalog/academic-regulations/index.html\#computerreq}
\item
  Disruptive classroom behavior:
  \url{http://www.fresnostate.edu/academics/facultyaffairs/documents/apm/419.pdf}
\end{itemize}

\hypertarget{university-services}{%
\section{University Services}\label{university-services}}

\begin{itemize}
\tightlist
\item
  \href{http://fresnostateasi.org/}{Associated Students, Inc.}
\item
  \href{http://fresnostate.edu/studentaffairs/dsc/index.html}{Dream
  Success Center}
\item
  \href{http://fresnostate.edu/studentaffairs/lrc}{Learning Center
  Information}
\item
  \href{https://www.fresnostate.edu/studentaffairs/health/}{Student
  Health and Counseling Center}
\item
  \href{http://www.fresnostate.edu/artshum/writingcenter/}{Writing
  Center}
\end{itemize}
