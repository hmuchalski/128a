Welcome to CHEM 129B, Organic Chemistry Laboratory 2! This document
contains all the information you need to know about the course.
\textbf{Read this document carefully, familiarize yourself with how the
course works, and maintain that familiarity throughout the term.} Almost
all questions about the course that you might ask can be answered by
referencing the syllabus.

This syllabus and schedule are subject to change in the event of
extenuating circumstances. If you are absent from class, it is your
responsibility to check on announcements made while you were absent.
Changes and corrections will be announced on Canvas.

\hypertarget{general-information}{%
\section{General information}\label{general-information}}

\begin{itemize}
\tightlist
\item
  \textbf{Course name and number}: CHEM 129B (2 Units)
\item
  \textbf{Prerequisites}:CHEM 129A is a prerequisite. CHEM 128B is a
  prerequisite or corequisite.
\item
  \textbf{Instructor}: Hubert Muchalski, PhD

  \begin{itemize}
  \tightlist
  \item
    phone: (559)-278-2711\footnote{This is my campus office number that
      redirects to my personal cell phone. Calling me is the best way to
      reach me. Sometimes, a 3-minute phone call resolves an issue
      better than multiple email exchanges.}
  \item
    email:
    \href{mailto:hmuchalski@mail.fresnostate.edu}{\nolinkurl{hmuchalski@mail.fresnostate.edu}})\footnote{Please
      note that I typically check email between 11 am and 6 pm, Monday
      through Friday. Usually, my response time is \emph{within 12 hours
      of reading the message}. We also have online course tools where
      you can ask questions to the entire class at any time, making it
      more likely to get a quick response.}
  \item
    office hours: TBA
  \end{itemize}
\end{itemize}

If there are questions or concerns that you have about this course that
you are not able to resolve with your instructor, please contact the
course coordinator Dr.~Hubert Muchalski. The the issue cannot be
resolved with the lab coordinator, please contact the department and/or
college administrators:

\begin{itemize}
\tightlist
\item
  Dr.~Joy Goto (Chair, Chemistry Department): jgoto@mail.fresnostate.edu
\item
  Chemistry Department Office: (559) 278-2103
\item
  Dr.~Christopher Meyer (Dean, College of Science and Mathematics):
  cmeyer@mail.fresnostate.edu
\end{itemize}

\hypertarget{what-is-chem-129b}{%
\section{What is CHEM 129B}\label{what-is-chem-129b}}

CHEM 129B, Organic Chemistry Laboratory 2, is the second part of the
year laboratory sequence in organic chemistry. As such it is primarily
concerned with introducing intermediate level concepts and techniques
used in organic chemistry. Although many of the techniques familiar to
you from the first semester lab will be used, additional ones will be
introduced including NMR, multi-step syntheses, green chemistry, and
introduction to the chemical literature and research through
project-based experiments. There will also be more emphasis on problem
solving, the application of theory, and structure identification via IR
and NMR spectroscopy.

\hypertarget{course-learning-outcomes}{%
\section{Course Learning Outcomes}\label{course-learning-outcomes}}

This course is organized into Modules and aims to help students achieve
learning outcomes through online activities. Each module contains
materials you need to complete the associated activities. Each
course-level student learning outcome is aligned with department-level
learning outcomes which are modeled after the standards set by the
American Chemical Society.

Upon completion of this course students will be able to:

\begin{enumerate}
\def\labelenumi{\arabic{enumi}.}
\tightlist
\item
  Communicate the structure and properties of organic molecules using
  common drawing and naming conventions
\item
  Analyze chemical structures and reaction conditions to make and defend
  predictions about chemical transformations
\item
  Use online databases to find relevant research articles containing
  information such as physical and chemical properties of organic
  molecules, synthetic procedures, and spectroscopic data.
\item
  Use software tools such as ChemDraw to draw chemical structures,
  reactions, and mechanisms
\item
  Plan a synthesis experiment by evaluating the information found in
  online databases and research articles
\item
  Analyze the results of an experiment and be able to identify sources
  of error and suggest improvements;
\item
  Interpret spectroscopic data of organic compounds to confirm the
  structure of organic compounds
\item
  Communicate the results of experiments to the instructor and peers in
  a written form (lab report)
\item
  Communicate the results of experiments to the instructor and peers in
  an oral or poster presentation
\end{enumerate}

\hypertarget{department-student-learning-outcomes}{%
\section{Department Student Learning
Outcomes}\label{department-student-learning-outcomes}}

\begin{enumerate}
\def\labelenumi{\arabic{enumi}.}
\tightlist
\item
  Students will apply their understanding of terminology, concepts,
  theories, and skills to solve problems by defining problems and
  research questions clearly, formulating testable hypotheses, designing
  and conducting experimental tests of hypotheses, analyzing and
  interpreting data, and drawing appropriate conclusions within
  professional ethical guidelines. (ACS Standards 7.1 \& 7.6)
\item
  Students will demonstrate the ability to conduct laboratory work of
  high quality including handling chemicals and other laboratory hazards
  in a safe, ethical, and socially responsible manner, keeping accurate,
  clear, concise, and complete records of their laboratory work in a
  notebook, properly using standard laboratory equipment and
  instruments, and evaluating the reliability and significance of
  laboratory data, all within professional ethical guidelines. (ACS
  Standards 7.1, 7.3, 7.6)
\item
  Students will complete a literature search in one or more of the five
  chemical sub disciplines by using common literature search techniques
  and tools to find recent journal articles from the peer-reviewed
  literature, critically read these articles to extract relevant
  information, and communicate the significance of these articles in
  written or oral formats within professional ethical guidelines. (ACS
  Standards 7.2 \& 7.6)
\item
  Students will demonstrate the ability to clearly and effectively
  communicate their scientific results and opinions using written
  formats while following professional style and format conventions
  within professional ethical guidelines. (ACS Standards 7.4 \& 7.6)
\item
  Students will demonstrate the ability to clearly and effectively
  communicate their scientific results and opinions using oral formats
  while following professional style and format conventions within
  professional ethical guidelines. (ACS Standards 7.4 \& 7.6)
\item
  Students will demonstrate the ability to function effectively in
  collaborative and group work environments including the ability to
  work on a component of a larger project and connect work with previous
  results within professional ethical standards. (ACS Standard 7.5 \&
  7.6)
\end{enumerate}

\hypertarget{course-materials-and-technology}{%
\section{Course materials and
technology}\label{course-materials-and-technology}}

\begin{itemize}
\tightlist
\item
  \textbf{Techniques Reference Manual} There are several free or
  affordable options here. See Canvas for the full list of resources
\item
  \textbf{Canvas:} The central repository for all course materials and
  information is our Canvas site, accessible through
  \url{https://fresnostate.instructure.com/courses/24394}. The Canvas
  site will house assignments, grades, and links to materials/resources.
\item
  \textbf{Personal computer:} You will need a x86 class personal
  computer, either Windows of macOS that can run desktop applications.
  Mobile devices have potential to augment the learning experience, but
  are not capable to run the apps we will use. Refer to Canvas for the
  computer and software requirements.
\item
  \textbf{Zoom:} Virtual class meetings will be held via Zoom. Links and
  passwords to zoom meetings will be published on Canvas.
\item
  \textbf{Document scanning}: Many assignments in this course are
  designed to be prepared by hand on paper. Few people own document
  scanners nowadays, but a mobile device with a scanning app can do a
  sufficient job at converting paper documents into PDFs. There are
  number of options available for both iOS and Android. Find one that
  you like and learn how to use it.
\item
  \textbf{Office 365 Apps}: Many assignments in this course are
  submitted as Word documents and are based on templates that are Word
  documents. Office 365 apps are available for Fresno State students.
  The suite includes the scanning app \emph{Office Lens}.
  \href{https://help.fresnostate.edu/students/software/office365.php}{Learn
  more}.
\end{itemize}

\hypertarget{course-modules}{%
\section{Course Modules}\label{course-modules}}

\begin{itemize}
\tightlist
\item
  M0. Start Here: Course Orientation
\item
  M1. Chemical Hazards and Risk Assessment
\item
  M3. Research Tools: SciFinder Scholar
\item
  M2. Research Tools: Journals and Scientific Databases
\item
  M4. Green Chemistry
\item
  M5. Spectroscopy part 1: IR \& MS
\item
  M6. Spectroscopy part 2: NMR
\item
  M7. Synthesis of Methyl Diantilis
\item
  M8. Greener Synthesis of Benzil
\item
  M9. Structural Elucidation of Acetophenone Derivatives
\item
  M10. Stereoselectivity of Hydride Reduction 1,2-Diketones
\item
  M11. Independent Project
\end{itemize}

\hypertarget{assignments-and-grading}{%
\section{Assignments and grading}\label{assignments-and-grading}}

Graded assignments are organized into assignment groups and each group
contributes to the final grade. Points earned for assignments in
different categories are not equivalent and should not be treated as
such. For example, it may take more work and effort to earn 50 points
for a report than 100 points for completing a simulation.

\begin{longtable}[]{@{}ll@{}}
\toprule
Assignment Category & \%Weight \\ \addlinespace
\midrule
\endhead
Chemical Safety & 10\% \\ \addlinespace
Reading assignments {[}Perusall{]} & 7\% \\ \addlinespace
Pre-lab assignments & 15\% \\ \addlinespace
Quizzes & 8\% \\ \addlinespace
Homework & 15\% \\ \addlinespace
Spectroscopy assignments & 20\% \\ \addlinespace
Reports & 15\% \\ \addlinespace
Independent Project & 10\% \\ \addlinespace
Extra credit & 3\% \\ \addlinespace
Total & 103\% \\ \addlinespace
\bottomrule
\end{longtable}

\newpage

Final grade will be determined based on overall performance according to
the weights in the table above.

\begin{longtable}[]{@{}ll@{}}
\toprule
Grade & Total Score \\ \addlinespace
\midrule
\endhead
A & 90--100\% \\ \addlinespace
B & 80--89\% \\ \addlinespace
C & 70--79\% \\ \addlinespace
D & 60-69\% \\ \addlinespace
F & \textless60\% \\ \addlinespace
\bottomrule
\end{longtable}

\hypertarget{virtual-classroom-policies-and-statements}{%
\section{Virtual classroom policies and
statements}\label{virtual-classroom-policies-and-statements}}

\hypertarget{intellectual-property}{%
\subsection{Intellectual Property}\label{intellectual-property}}

As part of your participation in virtual/online instruction, please
remember that the same student conduct rules that are used for in-person
classrooms instruction also apply for virtual/online classrooms.
Students are prohibited from any unauthorized recording, dissemination,
or publication of any academic presentation, including any online
classroom instruction, for any commercial purpose. In addition, students
may not record or use virtual/online instruction in any manner that
would violate copyright law. Students are to use all online/virtual
instruction exclusively for the educational purpose of the online class
in which the instruction is being provided. Students may not re-record
any online recordings or post any online recordings on any other format
(e.g., electronic, video, social media, audio recording, web page,
Internet, hard paper copy, etc.) for any purpose without the explicit
written permission of the faculty member providing the instruction.
Exceptions for disability-related accommodations will be addressed by
Services for Students with Disabilities (SSD) working in conjunction
with the student and faculty member.

In addition to course policies, you are expected to be familiar with
Academic Regulations described in the
\href{http://www.fresnostate.edu/catalog/academic-regulations/}{University
Catalog} as well as policies listed below.

\hypertarget{health-screening}{%
\subsection{Health screening}\label{health-screening}}

Students who come to campus for face-to-face classes will be required to
complete a daily health screening which will include temperature checks.
If you have experienced COVID-19 symptoms and/or have tested positive
within the past 10 days; or if you have had close contact (less than 6
feet for longer than 15 minutes while unmasked) with a suspected or
confirmed COVID-19 patient within the past 14 days, you are not allowed
to come to campus. Please complete the campus
\href{https://fresnostate.co1.qualtrics.com/jfe/form/SV_3faIAsuC8CzuFjD?Q_FormSessionID=FS_UFJ902LXgDJbKeZ}{online
reporting form}. A campus official will reply to provide guidance and
information.

\hypertarget{safety-measures}{%
\subsection{Safety Measures}\label{safety-measures}}

Consistent with the Governor's order and updated state public-health
guidelines, face masks or cloth face coverings are required to be worn
in public spaces on-campus and during in-person classes to reduce
possible exposure to COVID-19 and prevent the spread of the virus.
Physical distancing must be practiced by maintaining 6 feet of distance
between individuals. Good hygiene of hand washing for a minimum of 20
seconds or using hand sanitizer is required. Please avoid touching your
face with unclean hands. Disposable face masks will be provided to
anyone who arrives to campus without one. Please see university website
for the most updated information:
\href{http://www.fresnostate.edu/president/coronavirus/index.html}{www.fresnostate.edu/coronavirus}

\hypertarget{course-policies}{%
\section{Course policies}\label{course-policies}}

\hypertarget{attendance-and-participation}{%
\subsection{Attendance and
participation}\label{attendance-and-participation}}

Students are expected to attend and actively participate in all class
sessions. During the Zoom calls, students are expected to engage and
collaborate with peers on assigned tasks. Outside of the synchronous
class time, students are expected to complete reading assignments
(Perusall) to prepare for the class and help the instructor shape the
lesson plan.

The course attendance policy follows university's
\href{http://www.fresnostate.edu/academics/facultyaffairs/documents/apm/232.pdf}{APM
232: Policy on Student Absence}.

\hypertarget{late-workmake-up-policy}{%
\subsection{Late work/make-up policy}\label{late-workmake-up-policy}}

It is important that you understand my homework policy. I don't want to
weigh the different reasons students had for turning in late work using
my own social filters because it is unfair. As adult learners, sometimes
we must pay more attention to our lives than our schoolwork. That's OK;
I don't think students who do that are bad students. If the grade is
important to you, somehow you will find a way to get the stuff turned in
on time. No single missed weekly assignment will cause you to fail this
class.

I understand that life gets in the way. I also strongly believe in
revision and refinement of academic work and have developed the
following late work policy:

\begin{enumerate}
\def\labelenumi{\arabic{enumi}.}
\tightlist
\item
  Homework assignments are due on Sunday at 11:59 pm and at 12:00 am
  (Monday) a 5-day grace period begins.
\item
  Assignments received by the Sunday deadline will be graded first and
  returned as soon as possible. During the grace period you can
  revise/resubmit returned assignments\footnote{The assignment needs to
    be complete in order to qualify for the revision and resubmission.
    In other words, an incomplete assignment submitted just to meet the
    deadline will not be allowed to be revised/resubmitted. In such
    case, it is better to complete the assignment and turn it in within
    the 5-day grace period.} based on the feedback.
\item
  During the grace period you can still turn in the assignment without
  penalty, no questions asked, but it will not be graded in time for you
  to revise/resubmit based on the feedback.
\item
  Revision and resubmission is not available for assignments that aim to
  prepare you for the class meeting: pre-lab assignments and Perusall
  reading assignments. Those deadlines are final.
\item
  Assignments received after the grace period will not be accepted for a
  grade unless the reason meets the guidelines set forth by the
  university policy
  \href{http://www.fresnostate.edu/academics/facultyaffairs/documents/apm/232.pdf}{APM
  232: Policy on Student Absence}
\end{enumerate}

\hypertarget{technology-issues-when-submitting-work}{%
\subsection{Technology issues when submitting
work}\label{technology-issues-when-submitting-work}}

For assignments submitted electronically, it is your responsibility to
make sure they are submitted on time, through any means necessary, even
if technology issues arise. If a tech issue arises, it is your
responsibility to find another way to get it to the instructor (for
example, via an email attachment). Technology issues that are avoidable
or resolved with a simple work-around will not be considered valid
grounds for a deadline extension. For example, if you are trying to
upload a Lab to Canvas and Canvas won't accept the file, you should try
again later or send the file as an email attachment until you can upload
it successfully.

\hypertarget{respondus-lockdown-browser}{%
\subsection{Respondus Lockdown
Browser}\label{respondus-lockdown-browser}}

Respondus Lockdown Browser is a custom browser that locks down the
testing environment within Canvas. When students use the Respondus
Lockdown Browser they are unable to print, copy, go to a URL, or access
other applications. When an assessment is started, students are locked
into it until they submit it for grading. Available for both Windows and
Mac. Respondus Lockdown Browser does not work on a Chromebook.

Respondus Lockdown Browser uses a standard Windows or Mac installer that
can be downloaded by faculty or students from the following link (note:
this link is unique for Fresno State):

\url{http://www.respondus.com/lockdown/download.php?id=749643058}

\hypertarget{academic-dishonesty}{%
\subsection{Academic Dishonesty}\label{academic-dishonesty}}

For most assignments you are allowed and encouraged to work with others.
However, the final product that you submit for feedback must be the
result of your own efforts. Therefore you may share ideas and strategies
with others, but collaboration on the actual finished product you submit
is not allowed. Your work is expected to be the product of your own
thinking, written and explained in your own words with no parts of the
work copied from external sources such as books or websites, and done
clearly enough in your own mind that you could explain the work from
start to finish if asked. Specifically, this excludes:

\begin{itemize}
\tightlist
\item
  copying work from another student;
\item
  copying work from a website;
\item
  paraphrasing work done by another student or from print or Internet
  resources---i.e.~putting it in your own words---without coming up with
  the main ideas and strategies yourself; and
\item
  \emph{allowing or enabling} another student to copy or paraphrase work
  that you did, even if you did the original work yourself.
\end{itemize}

Violation of this policy is considered ``academic dishonesty'' and
carries with it strong punitive measures mandated by Fresno State,
including possible automatic failure of the course or suspension from
the university. For details, please see APM 235 by going to
\url{http://www.fresnostate.edu/aps/documents/apm/235.pdf}.

You may feel tempted to academic dishonesty at some point in the
semester. The work can be difficult, and many of you are under a lot of
stress. If you are considering academic dishonesty, please STOP, take a
breath, and remember that your classmates and I want you to succeed in
the course. You are not alone, and you have a strong network in the
class for getting help. The revision and resubmission policies mean that
it's OK to turn in work that isn't perfect. There is no need to be
academically dishonest! Just do your best on the work, and you'll have
the chance to revise it later.

\hypertarget{dropping-the-course-after-the-census-date}{%
\subsection{Dropping the course after the census
date}\label{dropping-the-course-after-the-census-date}}

A \emph{serious and compelling reason} is defined as an unexpected
condition that is not present prior to enrollment in the course that
unexpectedly arises and interferes with a student's ability to attend
class meetings and/or complete course requirements. The reason must be
acceptable to and verified by the instructor of record and the
department chair. The condition must be stated in writing on the
appropriate form. The student must provide documentation that
substantiates the condition.

Failing or performing poorly in a class is not an acceptable ``serious
and compelling reason'' within the University policy, nor is
dissatisfaction with the subject matter, class or instructor.

\hypertarget{university-policies-and-disclaimers}{%
\section{University policies and
disclaimers}\label{university-policies-and-disclaimers}}

\textbf{Students with Disabilities}: Upon identifying themselves to the
instructor and the university, students with disabilities will receive
reasonable accommodation for learning and evaluation. For more
information, contact Services to Students with Disabilities in the Henry
Madden Library, Room 1202 (278-2811).

\begin{itemize}
\tightlist
\item
  Class Schedule Policies:
  \url{http://fresnostate.edu/studentaffairs/classschedule/policy/}
\item
  Copyright Policy: \url{http://libguides.csufresno.edu/copyright}
\item
  Students with Disabilities:
  \url{http://fresnostate.edu/studentaffairs/careers/students/interests/disabilities.html}
\item
  Academic Integrity and Honor Code:
  \url{http://www.fresnostate.edu/academics/facultyaffairs/documents/apm/236.pdf}
\item
  Policy on Cheating and Plagiarism:
  \url{http://fresnostate.edu/studentaffairs/studentconduct/policies/cheating-plagiarism.html}
\item
  Add/Drop Course:
  \url{http://www.fresnostate.edu/studentaffairs/registrar/registration/}
\item
  Computer requirements:
  \url{https://www.fresnostate.edu/catalog/academic-regulations/index.html\#computerreq}
\item
  Disruptive classroom behavior:
  \url{http://www.fresnostate.edu/academics/facultyaffairs/documents/apm/419.pdf}
\end{itemize}

\hypertarget{university-services}{%
\section{University Services}\label{university-services}}

\begin{itemize}
\tightlist
\item
  \href{http://fresnostateasi.org/}{Associated Students, Inc.}
\item
  \href{http://fresnostate.edu/studentaffairs/dsc/index.html}{Dream
  Success Center}
\item
  \href{http://fresnostate.edu/studentaffairs/lrc}{Learning Center
  Information}
\item
  \href{https://www.fresnostate.edu/studentaffairs/health/}{Student
  Health and Counseling Center}
\item
  \href{http://www.fresnostate.edu/artshum/writingcenter/}{Writing
  Center}
\end{itemize}
