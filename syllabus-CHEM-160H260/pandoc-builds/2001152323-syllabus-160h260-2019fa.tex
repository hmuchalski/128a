\hypertarget{general-information}{%
\subsection{General information}\label{general-information}}

\begin{itemize}
\tightlist
\item
  \textbf{Course name and number}: Advanced Research Techniques
  (graduate, 3 Units); Research Techniques; CHEM 160H (undergraduate, 3
  Units).
\item
  \textbf{Time and location}: Friday, 08:00--08:50 AM in S1-242
  (lecture)
\item
  \textbf{Contact}: Science 1 room 352, phone (559) 278-2711,
  \url{hmuchalski@mail.fresnostate.edu}
\item
  \textbf{Canvas:} The central repository for all course materials and
  information is our Canvas site, accessible through
  \url{https://fresnostate.instructure.com/courses/19960}. The Canvas
  site will house your grades, links to handouts, videos, and other
  materials.
\item
  \textbf{Textbook:} The ACS Style Guide: Effective Communication of
  Scientific Information. Editor(s):Anne M. Coghill and Lorrin R.
  Garson.
  \href{http://login.hmlproxy.lib.csufresno.edu/login?url=http://dx.doi.org/10.1021/bk-2006-STYG}{DOI:
  10.1021/bk-2006-STYG}
\item
  \textbf{Software and Services:} Scifinder Scholar account, EndNote,
  and ChemDraw. Available for free for Fresno State students. Refer to
  instructions on how to obtain the software (Canvas).
\item
  \textbf{Office Hours:} I will be available for consultations after
  each class meeting (09:00--10:00 AM) and by appointment.
\end{itemize}

\hypertarget{introduction}{%
\subsection{Introduction}\label{introduction}}

\hypertarget{catalog-description}{%
\subsubsection{Catalog description}\label{catalog-description}}

Advanced concepts in experimental design. Development of practical
research expertise and communication skills through the planning,
completion, and presentation (both written and oral) of a short
laboratory project.

Success as a scientific researcher requires a number of skills that are
not fully developed at the undergraduate level. In addition to
proficiency with the appropriate experimental techniques and
instrumentation, researchers must to be able to plan their time and
experiments independently, and think creatively to overcome the problems
that are inevitably encountered. Outside of the laboratory, researchers
must have good written and oral communication skills both to present
their work to others in the scientific community, and to effectively
discuss their studies with fellow researchers.

In this class you will have the opportunity to improve your research
skills. You will work on an independent research project that may be
(but does not have to be) your MS thesis project. You will learn key
elements for effective scientific writing and presentations. You will
gain experience in scientific writing and oral presentations through
various assignments during the semester, culminating in the writing of a
term paper on your research project. The ultimate goal of this course is
to help you to develop the research skills necessary to successfully
complete the MS degree.

\hypertarget{slo}{%
\subsubsection{Student Learning Outcomes}\label{slo}}

Students who successfully complete this course should be able to:

\begin{itemize}
\tightlist
\item
  gain proficiency in the advanced experimental techniques in their area
  required to carry out graduate-level research;
\item
  design, plan, and execute experiments to test scientific hypotheses;
\item
  use the research tools and databases to find and extract relevant
  information from research papers;
\item
  cite references appropriately and avoid plagiarism;
\item
  communicate scientific information in both formal presentations and
  informal discussions; and
\item
  write competently in the style of the appropriate scientific journals.
\end{itemize}

\hypertarget{topics}{%
\subsubsection{Topics}\label{topics}}

\begin{enumerate}
\def\labelenumi{\arabic{enumi}.}
\tightlist
\item
  Tools for research: databases, journals, software, and services
\item
  Components of scientific communication (manuscript and supporting
  information)
\item
  Hypothesis and research proposal development, experimental design
\item
  Publishing and peer-review
\item
  Scientific misconduct and plagiarism
\end{enumerate}

\hypertarget{types-of-graded-work}{%
\subsection{Types of graded work}\label{types-of-graded-work}}

\begin{enumerate}
\def\labelenumi{\arabic{enumi}.}
\tightlist
\item
  Research proposal
\item
  Presentation
\item
  Research report
\item
  Participation in in-class and online discussions
\end{enumerate}

To support your learning and development I will provide assignments
during the semester. Major assignments (listed above) will be graded.
Other, smaller assignments will be ungraded. However, they must be
submitted on time and meet all listed criteria to be deemed
satisfactory.

\hypertarget{lab-section}{%
\subsubsection{Lab section}\label{lab-section}}

At the beginning of the semester you must identify a faculty member in
the department who is willing to act as your research advisor (if you do
not already have one). Together, you will identify your research
project. For graduate students it can be the same as the MS thesis
project. Undergraduate students will work on their Honors thesis
project. You will be expected to work on the research project for a
minimum of six hours per week during the semester. You should discuss
lab hour requirements and expectations with your advisor. The grade for
this section of the course will be assigned in consultation with your
research advisor.

\hypertarget{final-letter-grade-scheme}{%
\subsubsection{Final letter grade
scheme}\label{final-letter-grade-scheme}}

The letter grade scale followed a pattern close to the following: A =
90--100, B 80--89, C 70--79; D 60--69; and F \textless60.

\begin{longtable}[]{@{}ll@{}}
\toprule
Category & Subtotal\tabularnewline
\midrule
\endhead
Laboratory (graded w/ advisor) & 40\%\tabularnewline
Drafts & 15\%\tabularnewline
In-class presentations & 10\%\tabularnewline
Final presentation & 10\%\tabularnewline
Final paper & 25\%\tabularnewline
\bottomrule
\end{longtable}

\hypertarget{course-policies}{%
\subsection{Course policies}\label{course-policies}}

\hypertarget{late-work-policy}{%
\subsubsection{Late work policy}\label{late-work-policy}}

Late assignments will will not be accepted and will receive score of 0.

\hypertarget{technology-issues-when-submitting-work}{%
\subsubsection{Technology issues when submitting
work}\label{technology-issues-when-submitting-work}}

For assignments submitted electronically, it is your responsibility to
make sure they are submitted on time, through any means necessary, even
if technology issues arise. If a tech issue arises, it is your
responsibility to find another way to get it to me (for example, via an
email attachment). Technology issues that are avoidable or resolved with
a simple work-around will not be considered valid grounds for a deadline
extension.

\hypertarget{academic-dishonesty}{%
\subsubsection{Academic dishonesty}\label{academic-dishonesty}}

For most assignments you are allowed and encouraged to work with others.
However, the final product that you submit for feedback must be the
result of your own efforts. Therefore you may share ideas and strategies
with others, but collaboration on the actual finished product you submit
is not allowed. Your work is expected to be the product of your own
thinking, written and explained in your own words with no parts of the
work copied from external sources such as books or websites, and done
clearly enough in your own mind that you could explain the work from
start to finish if asked. Specifically, this excludes:

\begin{itemize}
\tightlist
\item
  copying work from another student;
\item
  copying work from a website;
\item
  paraphrasing work done by another student or from print or internet
  resources---i.e.~putting it in your own words---without coming up with
  the main ideas and strategies yourself; and
\item
  \emph{allowing or enabling} another student to copy or paraphrase work
  that you did, even if you did the original work yourself.
\end{itemize}

Violation of this policy is considered ``academic dishonesty'' and
carries with it strong punitive measures mandated by Fresno State,
including possible automatic failure of the course or suspension from
the university. For details, please see APM 235 by going to
\url{http://www.fresnostate.edu/aps/documents/apm/235.pdf}.

You may feel tempted to academic dishonesty at some point in the
semester. The work can be difficult, and many of you are under a lot of
stress. If you are considering academic dishonesty, please STOP, take a
breath, and remember that your classmates and I want you to succeed in
the course. You are not alone, and you have a strong network in the
class for getting help.

\hypertarget{plagiarism-detection}{%
\subsubsection{Plagiarism Detection}\label{plagiarism-detection}}

The campus subscribes to Turnitin and the SafeAssign plagiarism
prevention service through Canvas, and you will need to submit written
assignments to Turnitin/SafeAssign. Student work will be used for
plagiarism detection and for no other purpose. The student may indicate
in writing to the instructor that he/she refuses to participate in the
plagiarism detection process, in which case the instructor can use other
electronic means to verify the originality of their work.
Turnitin/SafeAssign Originality Reports will be available for your
viewing.

\hypertarget{dropping-the-course-after-the-census-date}{%
\subsubsection{Dropping the course after the census
date}\label{dropping-the-course-after-the-census-date}}

A \emph{serious and compelling reason} is defined as an unexpected
condition that is not present prior to enrollment in the course that
unexpectedly arises and interferes with a student's ability to attend
class meetings and/or complete course requirements. The reason must be
acceptable to and verified by the instructor of record and the
department chair. The condition must be stated in writing on the
appropriate form. The student must provide documentation that
substantiates the condition.

Failing or performing poorly in a class is not an acceptable ``serious
and compelling reason'' within the University policy, nor is
dissatisfaction with the subject matter, class or instructor.

\hypertarget{university-policies-and-disclaimers}{%
\subsection{University policies and
disclaimers}\label{university-policies-and-disclaimers}}

In addition to course policies, you are expected to be familiar with
Academic Regulations described in the
\href{http://www.fresnostate.edu/catalog/academic-regulations/}{University
Catalog} as well as policies listed below.

\textbf{Students with Disabilities}: Upon identifying themselves to the
instructor and the university, students with disabilities will receive
reasonable accommodation for learning and evaluation. For more
information, contact Services to Students with Disabilities in the Henry
Madden Library, Room 1202 (278-2811).

\begin{itemize}
\tightlist
\item
  Class Schedule Policies:
  \url{http://fresnostate.edu/studentaffairs/classschedule/policy/}
\item
  Copyright Policy: \url{http://libguides.csufresno.edu/copyright}
\item
  Students with Disabilities:
  \url{http://fresnostate.edu/studentaffairs/careers/students/interests/disabilities.html}
\item
  Academic Integrity and Honor Code:
  \url{http://www.fresnostate.edu/academics/facultyaffairs/documents/apm/236.pdf}
\item
  Policy on Cheating and Plagiarism:
  \url{http://fresnostate.edu/studentaffairs/studentconduct/policies/cheating-plagiarism.html}
\item
  Add/Drop Course:
  \url{http://www.fresnostate.edu/studentaffairs/registrar/registration/}
\item
  Computer requirements:
  \url{https://www.fresnostate.edu/catalog/academic-regulations/index.html\#computerreq}
\item
  Disruptive classroom behavior:
  \url{http://www.fresnostate.edu/academics/facultyaffairs/documents/apm/419.pdf}
\end{itemize}
