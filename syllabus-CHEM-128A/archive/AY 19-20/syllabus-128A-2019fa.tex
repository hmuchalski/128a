\newpage

Welcome to CHEM 128A, Organic Chemistry 1! This course will introduce
you to one of the richest and most beautiful areas of modern chemistry:
\emph{chemistry of carbon compounds}. In CHEM 128A we will learn skills
that are essential for understanding modern biochemistry, medicine, and
the chemical reactions related to life.

\textbf{Important note:} This course uses mastery-based grading system.
It is different than the one that you might be used to. The details are
explained in sections below so \emph{please read the syllabus
carefully.} It is nearly 3,300 words for a reason. Almost all questions
about the course that you might ask can be answered by referencing the
syllabus. If you are uncertain that you understand all rules and
regulations, please contact me.

\hypertarget{changelog}{%
\section{Changelog}\label{changelog}}

Until the census date, the content of this syllabus may change. The
schedule and procedures for this course are subject to change in the
event of extenuating circumstances. If you are absent from class, it is
your responsibility to check on announcements made while you were
absent. Changes and corrections are listed in the changelog below and
will be announced on Canvas.

\begin{itemize}
\tightlist
\item
  2019-08-14: Syllabus updated and published on Canvas
\item
  2019-08-22: Updated final exam date. Added office hours information.
  Corrected typographical errors.
\end{itemize}

\hypertarget{course-information}{%
\section{Course information}\label{course-information}}

\begin{itemize}
\tightlist
\item
  \textbf{Course name and number}: CHEM 128A (76318-07-LEC), 3 units
\item
  \textbf{Prerequisites}: Grade C or better in CHEM 1B or CHEM 8.
\item
  \textbf{Meetings}: Tu/Th 3:30--4:45 PM in S1-109
\item
  \textbf{Instructor office and phone}: Science 1 room 352; (559)
  278-2711
\item
  \textbf{Email}\footnote{Please note that I typically check email
    between 11 am and 5 pm, Monday through Friday. Usually, my response
    time is \emph{within 12 hours of reading the message}. We also have
    online course tools where you can ask questions to the entire class
    at any time, making it more likely to get a quick response.}:
  \url{hmuchalski@mail.fresnostate.edu} or
  \url{hmuchalski@csufresno.edu} (they go to the same inbox).
\item
  \textbf{Office Hours}: Walk-in office hours are Monday and Wednesday
  12:00--01:00 pm. Additional consultation appointments can be scheduled
  through calendar function ``Find Appointments'' on Canvas.
\end{itemize}

\hypertarget{course-description}{%
\section{Course description}\label{course-description}}

CHEM 128A is the first part of a two-semester course in organic
chemistry, the chemistry of carbon-containing compounds. Topics in this
course will be focused on the principles of bonding, structure,
reactivity, and synthesis of organic materials. Also, a significant
portion of this course will address the analytical techniques routinely
used by organic chemists in their research. Organic chemistry is central
to understanding multiple other disciplines. Lectures and problems will
often feature organic compounds and reactions in the context of biology,
pharmacy, medicine, materials, and energy science.

\hypertarget{course-goals-and-learning-objectives}{%
\subsection{Course goals and learning
objectives}\label{course-goals-and-learning-objectives}}

The goal of this course is that students reach fluency in communicating
structure and reactivity of orgnic molecules. This knowledge is
essential for appreciating the world and preparing for future
professional work. At the successful completion of this course you will
be able to:

\begin{itemize}
\tightlist
\item
  communicate the structure and properties of organic molecules using
  common drawing and naming conventions;
\item
  analyze chemical structures and reactions to make and defend
  predictions about chemical processes;
\item
  use curved arrow notation to draw mechanisms of chemical reactions;
\item
  propose a synthesis of an organic molecule;
\item
  connect the ideas of organic chemistry with your own personal and
  professional interests; and
\item
  demonstrate competence in self-regulated learning of new technical
  material, managing time and tasks, and educational interactions with
  peers.
\end{itemize}

\hypertarget{requirements}{%
\section{Requirements}\label{requirements}}

\hypertarget{course-materials-and-technology}{%
\subsection{Course materials and
technology}\label{course-materials-and-technology}}

This course uses Immediate Access to course materials. All students have
access to a digital version of the textbook and associated materials on
the first day of class and have until the 10th day of instruction to
OPT-OUT of the low cost digital materials, but will have to purchase the
materials elsewhere. Students are automatically charged on the 10th day
(5th day for Summer courses) to continue to have access to course
materials for the rest of term. See Canvas for details.

\begin{itemize}
\tightlist
\item
  \textbf{Canvas:} The central repository for all course materials and
  information found here: \url{https://fresnostate.instructure.com}.
\item
  \textbf{Textbook:} ``Organic Chemistry'' by David Klein 3rd edition
  published by Wiley (via Immediate Access)
\item
  \textbf{WileyPLUS with ORION:} Online learning platform and homework.
  WileyPLUS is integrated with Canvas and all links to assignments and
  materials will be posted on Canvas. \footnote{WileyPLUS version is
    tied to the edition of the textbook. If you opt out of Immediate
    Access and decide to buy a paper version of the textbook, make sure
    that your access code is for the 3rd edition.}
\item
  \textbf{Student response system:} Each student will need an
  \emph{i\textgreater clicker2} remote or access to the iClicker Reef
  mobile app. The officially supported model is
  \emph{i\textgreater clicker2} (available at the Bookstore). \footnote{You
    can borrow hardware remote from a friend or buy a used one. Remotes
    need to be registered online to be linked to your name. We will go
    through registration during first class meeting.}
\item
  \textbf{Study Guide and Solutions Manual} As the name suggests. This
  is optional item. It is included in IA.
\end{itemize}

\hypertarget{in-class-requirements}{%
\subsection{In-class requirements}\label{in-class-requirements}}

Five 90-minute meetings per week (Monday--Friday) where we review and
discuss concepts and practice skills. You'll be working with your
classmates to make sense of concepts and work on creative applications
of those basics through group problem-solving sessions, discussions
driven by interactive polling activities, and more. All of the work you
do in class is carefully designed to promote learning of the concepts of
the course (7.5 hours per week).

\hypertarget{out-of-class-requirements}{%
\subsection{Out-of-class requirements}\label{out-of-class-requirements}}

Prior to the class meeting, students are required to work actively to
get their first contact with new concepts by reading the textbook,
watching videos, and completing pre-class assignments. Students are also
expected to engage in online discussions and Q\&A (15-20 hours per
week).

\hypertarget{types-of-graded-work}{%
\section{Types of graded work}\label{types-of-graded-work}}

There are six types of assignments and tests that you will encounter in
this course:

\begin{enumerate}
\def\labelenumi{\arabic{enumi}.}
\tightlist
\item
  \textbf{ORION diagnostics.} These are low-stakes online assignments
  assessing the knowledge you gained through self-guided learning
  (reading assignments and Skill Builder modules). These assignments are
  graded the basis of completeness and effort (not correctness).
\item
  \textbf{Mastery assignments.} Online homework (WileyPLUS) designed to
  build expertise in newly introduced concepts.
\item
  \textbf{Learning Target Assessments (LTAs).} Short quizzes on one of
  the 11 Learning Targets in the course. Each is graded
  \emph{Satisfactory} or \emph{Progressing} on the basis of correctness
  and completeness.
\item
  \textbf{Participation Credits.} Taking part in student response polls
  in class.
\item
  \textbf{Connections \& Synthesis.} Take-home assignments that explore
  organic chemistry applications involving real life examples from the
  lab and beyond. \emph{Connections} are graded \emph{Satisfactory},
  \emph{Progresing}, or \emph{Incomplete} on correctness, clarity, and
  completeness.
\item
  \textbf{Final Exam.} This exam measures retention of knowledge and
  skills you learned throughout the semester. The exam is comprehensive
  (all the material covered throughout the semester) and is composed of
  70 multiple choice questions designed by experts from the American
  Chemical Society. Final exam will be on \textbf{Thursday, December
  19th 05:45--07:45 PM}.
\end{enumerate}

\hypertarget{pre-class-readings-and-diagnostics}{%
\subsection{Pre-class readings and
diagnostics}\label{pre-class-readings-and-diagnostics}}

This class is designed according to a model in which pre-class
activities provide you with a structured introduction to the new
material so that we can review, discuss, and practice during in-class
meetings. After you read the chapter and complete guided exercises, you
will be prompted to complete ORION assignment. ORION is an adaptive
learning tool which tests the limits of your knowledge. It will always
try to ask you questions that are appropriate for your current level of
proficiency.

Completing the pre-class assignments serves two roles. First, it gives
you an idea about the level of mastery you achieved through self-guided
learning. Second, it helps me decide what activities to plan and what
concepts to focus on in class. Please note that \textbf{for pre-class
ORION questions correctness is not factored into the grade.} You should
feel free to give your best effort on each question without fear of
being counted off for wrong answers. That being said, it pays to learn
as much as you can on your own to get the ORION proficiency score on a
good start.

\hypertarget{mastery-homework}{%
\subsection{Mastery homework}\label{mastery-homework}}

At the end of each module you will practice by completing more online
assignments. \emph{WileyPLUS Mastery} assignments are typically 10
questions on a particular topic and require that you attempt each of the
questions and get a total percentage correct. You can attempt each
question only once. However, if you do not get a percentage correct to
show mastery (\textgreater75\%), you can reset the assignment and try
again. Resetting the assignment will generate a new set of questions for
you to attempt. Your best attempt at this assignment will be your final
recorded score. Resetting the assignment will not change the difficulty
level of the questions. 75\% or more correct to receive 10 points; 50\%
or more correct to receive 5 points; less than 50\% correct to receive 0
points.

\hypertarget{learning-target-assessments-ltas}{%
\subsection{Learning target assessments
(LTAs)}\label{learning-target-assessments-ltas}}

The content and the skills you will learn in the course are divided
\protect\hyperlink{LT}{into Learning Targets}. There are six (6)
\emph{Essential Learning Targets} related to the core skills and
knowledge. Showing mastery of ELTs is required to pass the course. The
remaining 5 are \emph{Supplemental Learning Targets} that focus on
additional important skills from organic chemistry. Mastery is
demonstrated by completing short quizzes, each addressing a single
Learning Target. The quizzes, called Learning Target Assessments, or
LTAs, are graded either \emph{satisfactory} or \emph{progressing}. What
constitutes \emph{satisfactory} or \emph{progressing} work will be
spelled out explicitly for each Learning Target and made known to you in
advance.

Some in-class time will be devoted to taking LTAs. Initial attempt to
pass an LTA during in-class session is mandatory for all students.
Re-take can be ``purchased'' with tokens (\protect\hyperlink{tokens}{see
details below}). There is 20 minute limit on all LTAs.

\hypertarget{participation-credits}{%
\subsection{Participation credits}\label{participation-credits}}

A participation credit will be awarded for participation in iClicker
questions in class but can also be awarded for asking an insightful
question in class or during office hours. Research shows that student
response systems (clickers) help students learn more and do better in
the course. I have successfully used the student response system to
gauge student learning and direct the flow of the lecture. Student
polling will be used in almost every lecture and students must respond
to at least 75\% of questions on a particular day to earn PC for
participation in the session.

\hypertarget{connections-synthesis}{%
\subsection{Connections \& Synthesis}\label{connections-synthesis}}

Students who aspire to receive higher grade in this course and/or
consider using me as a reference for their graduate/professional school
applications will be asked to earn \emph{satisfactory} grade on
additional take-home assignments. \emph{Connections} is a written
assignment that require critical analysis and evaluation of information
in the context of the knowledge you learned in the course. For the
\emph{Synthesis} assignment you will be asked to design a synthesis of a
target molecule of my choice.

\hypertarget{how-is-the-letter-grade-determined}{%
\section{How is the letter grade
determined}\label{how-is-the-letter-grade-determined}}

I use a mastery-based grading system that is designed to provide you
with control over the grading process. Final letter grade in CHEM 128A
will be determined by the quantity and quality of evidence you can
provide that you have mastered the concepts of the course. \textbf{You
will have multiple attempts to earn a \emph{satisfactory} grade} on most
assignments. The grading system in this CHEM 128A course allows
revisions and multiple attempts to demonstrate a satisfactory level of
learning. Grades on LTAs are not final until the end of the semester and
can be attempted again. Read \protect\hyperlink{revisions}{Revision and
reassessment} policy for more details.

The grade you earn at the end of the semester is determined by referring
to the list below. There will be no statistical or numerical adjustments
(a.k.a. grading on a curve). All items within ``grade bundle'' must be
completed to receive the letter grade. Failing grade (F) is given if not
all the requirements for a ``D'' are met.

\hypertarget{d-grade-bundle}{%
\subsection{D grade bundle}\label{d-grade-bundle}}

\begin{itemize}
\tightlist
\item[$\square$]
  5 ELTAs
\item[$\square$]
  15 points on the final exam
\item[$\square$]
  ORION diagnostics for 4 chapters
\item[$\square$]
  50\% on Mastery
\end{itemize}

\hypertarget{c-grade-bundle}{%
\subsection{C grade bundle}\label{c-grade-bundle}}

\begin{itemize}
\tightlist
\item[$\square$]
  6 ELTAs
\item[$\square$]
  22 points on the final exam
\item[$\square$]
  ORION diagnostics for 6 chapters
\item[$\square$]
  50\% on two the following: ORION proficiency, Mastery, iClicker
\end{itemize}

\hypertarget{b-grade-bundle}{%
\subsection{B grade bundle}\label{b-grade-bundle}}

\begin{itemize}
\tightlist
\item[$\square$]
  6 ELTAs
\item[$\square$]
  2 SLTAs
\item[$\square$]
  29 points on the final exam
\item[$\square$]
  ORION diagnostics for 8 chapters
\item[$\square$]
  60\% on two of the following: ORION proficiency, Mastery, iClicker
\item[$\square$]
  One of the following: Connections, Synthesis, 36 points on the final
  exam
\end{itemize}

\hypertarget{a-grade-bundle}{%
\subsection{A grade bundle}\label{a-grade-bundle}}

\begin{itemize}
\tightlist
\item[$\square$]
  6 ELTAs
\item[$\square$]
  4 SLTAs
\item[$\square$]
  36 points on the final exam
\item[$\square$]
  ORION diagnostics for 10 chapters
\item[$\square$]
  70\% on two of the following: ORION proficiency, Mastery, iClicker
\item[$\square$]
  Two of the following: Connections, Synthesis, 43 points on the final
  exam
\end{itemize}

\hypertarget{revisions}{%
\section{Revision and reassessment}\label{revisions}}

The grading system in our course insists that you show consistent
excellence in all assignments in the course---outstanding work on
quizzes, for example, does not ``bring up'' poor work on online
assignments. This can be challenging, but the course also provides a
robust system of revision and reassessment for most graded tasks, so
that if you aren't happy with a grade on an assignment, you'll have
multiple chances to try again or fix any mistakes.

\hypertarget{revision-of-wileyplus-assignments}{%
\subsection{Revision of WileyPLUS
assignments}\label{revision-of-wileyplus-assignments}}

When you submit WileyPLUS assignment, you receive instant feedback on
which answers were right and which ones were wrong. You may reattempt
any online homework set as many times as you want until the deadline for
the set. After the deadline, no revision is allowed and your score is
final.

\hypertarget{revision-of-learning-target-assessments-ltas}{%
\subsection{Revision of Learning Target Assessments
(LTAs)}\label{revision-of-learning-target-assessments-ltas}}

LTAs that receive a \emph{Progressing} grade may be reattempted. Each
additional attempt will cover the same material but will not be
identical to past quizzes.

I found that students tend to defer retaking LTAs until it's too late.
Therefore, each student can attempt maximum 2 LTA re-takes per week and
LTA retakes will be more costly as time goes by
(\protect\hyperlink{lta-cost}{see Appendix A}). LTA re-take sessions
schedule will be announced on Canvas. Also, I need to know in advance
which LTAs you plan to re-attempt. Thus, you must fill an online form no
later than 12 pm the day you plant to take the quiz. To fill out the
form go to this page: \url{https://forms.gle/vzeBem1uMybtgb7B7}

\hypertarget{tokens}{%
\subsection{Tokens}\label{tokens}}

Tokens are a ``currency'' in the course that you can use to purchase LTA
re-takes and exceptions to some course rules. Each student begins the
course with 20 tokens which can be exchanged for:

\begin{itemize}
\tightlist
\item
  LTA re-takes according to the policy describe above;
\item
  feedback on a draft of \emph{Connections} or \emph{Synthesis}
  assignment (2 tokens/consultation);
\item
  24-hour deadline extension on Mastery assignments in a module;
\item
  Participation Credits (2 tokens/credit);
\item
  total reset of ORION proficiency metric (3 tokens); and
\item
  points on the final exam (3 tokens/points, max 5 points).
\end{itemize}

\hypertarget{course-policies}{%
\section{Course policies}\label{course-policies}}

\hypertarget{technology-issues-when-submitting-work}{%
\subsection{Technology issues when submitting
work}\label{technology-issues-when-submitting-work}}

WileyPLUS ORION and Mastery assignments are submitted electronically. It
is the student's responsibility to make sure these items are submitted
on time, through any means necessary, even if technology issues arise.
Technology issues that are avoidable or resolved with a simple
work-around will not be considered valid grounds for a deadline
extension. For example, if you are trying to upload to Canvas and Canvas
won't accept the file, you should try again later or send the file as an
email attachment until you can upload it successfully.

\hypertarget{academic-dishonesty}{%
\subsection{Academic Dishonesty}\label{academic-dishonesty}}

Your work on Learning Target Assessments must be done individually, and
all collaboration is prohibited.

For most other assignments you are allowed and encouraged to work with
others. However, the final product that you submit for feedback must be
the result of your own efforts. Therefore you may share ideas and
strategies with others, but collaboration on the actual finished product
you submit is not allowed. Your work is expected to be the product of
your own thinking, written and explained in your own words with no parts
of the work copied from external sources such as books or websites, and
done clearly enough in your own mind that you could explain the work
from start to finish if asked. Specifically, this excludes:

\begin{itemize}
\tightlist
\item
  copying work from another student;
\item
  copying work from a website;
\item
  paraphrasing work done by another student or from print or internet
  resources---i.e.~putting it in your own words---without coming up with
  the main ideas and strategies yourself; and
\item
  \emph{allowing or enabling} another student to copy or paraphrase work
  that you did, even if you did the original work yourself.
\end{itemize}

Violation of this policy is considered ``academic dishonesty'' and
carries with it strong punitive measures mandated by Fresno State,
including possible automatic failure of the course or suspension from
the university. For details, please see APM 235 by going to
\url{http://www.fresnostate.edu/aps/documents/apm/235.pdf}.

You may feel tempted to academic dishonesty at some point in the
semester. The work can be difficult, and many of you are under a lot of
stress. If you are considering academic dishonesty, please STOP, take a
breath, and remember that your classmates and I want you to succeed in
the course. You are not alone, and you have a strong network in the
class for getting help. The revision and resubmission policies mean that
it's OK to turn in work that isn't perfect. There is no need to be
academically dishonest! Just do your best on the work, and you'll have
the chance to revise it later.

\hypertarget{lta-make-up-policy}{%
\subsection{LTA make-up policy}\label{lta-make-up-policy}}

If you know in advance that you will miss an LTA (first attempt), and
have a valid reason that can be verified by a document (e.g.~a doctor's
letter, or a letter from an athlete's sports team coach), I will decide
on an individual basis. Notify me as soon as you confirm that you will
not be able to take an LTA and I will arrange an alternative date/time
for you.

\hypertarget{dropping-the-course-after-the-census-date}{%
\subsection{Dropping the course after the census
date}\label{dropping-the-course-after-the-census-date}}

A \emph{serious and compelling reason} is defined as an unexpected
condition that is not present prior to enrollment in the course that
unexpectedly arises and interferes with a student's ability to attend
class meetings and/or complete course requirements. The reason must be
acceptable to and verified by the instructor of record and the
department chair. The condition must be stated in writing on the
appropriate form. The student must provide documentation that
substantiates the condition.

Failing or performing poorly in a class is not an acceptable ``serious
and compelling reason'' within the University policy, nor is
dissatisfaction with the subject matter, class or instructor.

\hypertarget{policy-on-children-in-class}{%
\subsection{Policy on children in
class}\label{policy-on-children-in-class}}

Currently, the university does not have a formal policy on children in
the classroom. The policy described here is thus, a reflection of my own
beliefs and commitments to student, staff and faculty parents. I hope
that you will feel comfortable disclosing your student-parent status to
me. This is the first step in my being able to accommodate any special
needs that arise. While I maintain the same high expectations for all
student in my classes regardless of parenting status, I am happy to
problem solve with you in a way that makes you feel supported as you
strive for school-parenting balance.

\hypertarget{university-policies-and-disclaimers}{%
\section{University policies and
disclaimers}\label{university-policies-and-disclaimers}}

In addition to course policies, you are expected to be familiar with
Academic Regulations described in the
\href{http://www.fresnostate.edu/catalog/academic-regulations/}{University
Catalog} as well as policies listed below.

\textbf{Students with Disabilities}: Upon identifying themselves to the
instructor and the university, students with disabilities will receive
reasonable accommodation for learning and evaluation. For more
information, contact Services to Students with Disabilities in the Henry
Madden Library, Room 1202 (278-2811).

\begin{itemize}
\item
  Class Schedule Policies:
  \url{http://fresnostate.edu/studentaffairs/classschedule/policy/}
\item
  Copyright Policy: \url{http://libguides.csufresno.edu/copyright}
\item
  Students with Disabilities:
  \url{http://fresnostate.edu/studentaffairs/careers/students/interests/disabilities.html}
\item
  Academic Integrity and Honor Code:
  \url{http://www.fresnostate.edu/academics/facultyaffairs/documents/apm/236.pdf}
\item
  Policy on Cheating and Plagiarism:
  \url{http://fresnostate.edu/studentaffairs/studentconduct/policies/cheating-plagiarism.html}
\item
  Add/Drop Course:
  \url{http://www.fresnostate.edu/studentaffairs/registrar/registration/}
\item
  Computer requirements:
  \url{https://www.fresnostate.edu/catalog/academic-regulations/index.html\#computerreq}
\item
  Disruptive classroom behavior:
  \url{http://www.fresnostate.edu/academics/facultyaffairs/documents/apm/419.pdf}
\item
\end{itemize}

\hypertarget{university-services}{%
\section{University Services}\label{university-services}}

\begin{itemize}
\tightlist
\item
  \href{http://fresnostateasi.org/}{Associated Students, Inc.}
\item
  \href{http://fresnostate.edu/studentaffairs/dsc/index.html}{Dream
  Success Center}
\item
  \href{http://fresnostate.edu/studentaffairs/lrc}{Learning Center
  Information}
\item
  \href{https://www.fresnostate.edu/studentaffairs/health/}{Student
  Health and Counseling Center}
\item
  \href{http://www.fresnostate.edu/artshum/writingcenter/}{Writing
  Center}
\end{itemize}

\hypertarget{lta-cost}{%
\section{Appendix A}\label{lta-cost}}

\begin{longtable}[]{@{}llllllllllll@{}}
\toprule
date & ELT1 & ELT2 & ELT3 & ELT4 & ELT5 & ELT6 & SLT1 & SLT2 & SLT3 &
SLT4 & SLT5\tabularnewline
\midrule
\endhead
09/03 & 0 & 0 & 0 & 0 & 0 & 0 & 0 & 0 & 0 & 0 & 0\tabularnewline
09/10 & 1 & 0 & 0 & 0 & 0 & 0 & 0 & 0 & 0 & 0 & 0\tabularnewline
09/17 & 2 & 0 & 0 & 0 & 0 & 0 & 0 & 0 & 0 & 0 & 0\tabularnewline
09/24 & 3 & 1 & 1 & 0 & 0 & 0 & 0 & 0 & 0 & 0 & 0\tabularnewline
10/01 & 4 & 2 & 2 & 1 & 0 & 0 & 0 & 0 & 0 & 0 & 0\tabularnewline
10/08 & 5 & 3 & 3 & 2 & 0 & 0 & 0 & 0 & 0 & 0 & 0\tabularnewline
10/15 & 6 & 4 & 4 & 3 & 1 & 0 & 0 & 0 & 0 & 0 & 0\tabularnewline
10/22 & 7 & 5 & 5 & 4 & 2 & 1 & 0 & 0 & 0 & 0 & 0\tabularnewline
10/29 & 8 & 6 & 6 & 5 & 3 & 2 & 0 & 0 & 0 & 0 & 0\tabularnewline
11/05 & 9 & 7 & 7 & 6 & 4 & 3 & 1 & 0 & 0 & 0 & 0\tabularnewline
11/12 & 10 & 8 & 8 & 7 & 5 & 4 & 2 & 1 & 0 & 0 & 0\tabularnewline
11/19 & 10 & 9 & 9 & 8 & 6 & 5 & 3 & 2 & 1 & 0 & 0\tabularnewline
11/26 & 10 & 10 & 10 & 9 & 7 & 6 & 4 & 3 & 2 & 1 & 1\tabularnewline
12/03 & 10 & 10 & 10 & 10 & 8 & 7 & 5 & 4 & 3 & 2 & 2\tabularnewline
12/10 & 10 & 10 & 10 & 10 & 9 & 8 & 6 & 5 & 4 & 3 & 3\tabularnewline
\bottomrule
\end{longtable}
