\hypertarget{changelog}{%
\section{Changelog}\label{changelog}}

This syllabus and schedule are subject to change in the event of
extenuating circumstances. If you are absent from class, it is your
responsibility to check on announcements made while you were absent.
Changes and corrections are listed in the changelog below and will be
announced on Canvas.

\begin{itemize}
\tightlist
\item
  2020-08-14: First draft published on Canvas
\end{itemize}

\newpage

\hypertarget{introduction}{%
\section{Introduction}\label{introduction}}

Welcome to the second semester of course in organic chemistry (CHEM
128B)! In this course we will continue to explore one of the richest and
most beautiful areas of modern chemistry: \emph{chemistry of
carbon-basaed compounds}. This course is required for majors in life and
physical sciences because it covers topics and concepts that are
essential for understanding of biochemistry and medicine.

\textbf{Important note about grading:} This course uses a different
grading method to one that you might be used to. The details are
explained in sections below. \emph{Read the syllabus carefully.} It is
nearly 5,000 words for a reason. Almost all questions about the course
that you might ask can be answered by referencing the syllabus. If you
are uncertain that you understand all rules and regulations, please
contact me.

\begin{itemize}
\tightlist
\item
  \textbf{Course name and number}: CHEM 128B (75500 02-LEC)
\item
  \textbf{Units}: 3
\item
  \textbf{Pre-requisite}: Passed CHEM 128A with grade ``C'' or better.
\item
  \textbf{Meetings}: MWF 9:00--9:50 am on Zoom
\item
  \textbf{Office Phone}: (559) 278-2711
\item
  \textbf{Email}: \url{hmuchalski@csufresno.edu} or
  \url{hmuchalski@mail.fresnostate.edu} (they go to the same inbox).
  Please note that I typically check email between 11 am and 5 pm,
  Monday thru Friday. Usually, my response time is \emph{within 12 hours
  of reading the message}. We also have online course tools where you
  can ask questions to the entire class at any time, making it more
  likely to get a quick response.
\item
  \textbf{Office Hours}: Appointments can be scheduled through calendar
  function ``Find Appointments'' on Canvas.
\end{itemize}

\hypertarget{requirements}{%
\section{Requirements}\label{requirements}}

\hypertarget{course-materials}{%
\subsection{Course materials}\label{course-materials}}

This course uses Immediate Access to course materials. All students have
access to a digital version of the textbook and associated materials on
the first day of class and have until the 10th day of instruction to
OPT-OUT of the low cost digital materials, but will have to purchase the
materials elsewhere. Students are automatically charged on the 10th day
(5th day for Summer courses) to continue to have access to course
materials for the rest of term. See Canvas for details. More info can be
found at \url{https://als.csuprojects.org/immediate_access_programs}

\begin{itemize}
\tightlist
\item
  \textbf{Canvas:} The central repository for all course materials and
  information found here: \url{https://fresnostate.instructure.com}.
\item
  \textbf{Textbook:} ``Organic Chemistry'' by David Klein 3rd edition
  published by Wiley (via Immediate Access)
\item
  \textbf{WileyPLUS with ORION:} Online learning platform and homework.
  WileyPLUS is integrated with Canvas and all links to assignments and
  materials will be posted on Canvas. \footnote{WileyPLUS version is
    tied to the edition of the textbook. If you opt out of Immediate
    Access and decide to buy a paper version of the textbook, make sure
    that your access code is for the 3rd edition.}
\item
  \textbf{Classroom Response System:} We will use iClicker Reef mobile
  app to collect responses to questions posed during class. Physical
  remotes do not work in a virtual setting.
\item
  \textbf{Study Guide and Solutions Manual} As the name suggests. This
  is optional item. It is included in IA.
\end{itemize}

\hypertarget{technology}{%
\subsection{Technology}\label{technology}}

To use the course tools, you will need to have access to the following:

\begin{itemize}
\tightlist
\item
  \textbf{A personal computer}: running Windows or macOS, that can run
  desktop applications and has a reliable access to high-speed internet.
  A tablet device is an alternative, but the online homework platform
  doesn't work as well on mobile devices.
\item
  \textbf{A modern web browser}: Chrome is preferred, but browsers such
  as Edge, Firefox, or Safari are also fine.
\item
  \textbf{Zoom:} Virtual class meetings will be held via Zoom. Links and
  passwords to zoom meetings will be published on Canvas.
\item
  \textbf{Document scanning tool}: Many assignments in this course are
  designed to be prepared by hand on paper. Few people own document
  scanners nowadays, but a mobile device with a scanning app can do a
  sufficient job at converting paper documents into PDFs. There are
  number of options available for both iOS and Android. Find one that
  you like and learn how to use it.
\item
  \textbf{Active Fresno State network account} so that you can access
  email, Canvas, and Google Suite
\end{itemize}

If you have any issue with accessing any of the above, please let me
know as soon as possible. We will use a variety of additional course
tools during the semester, but they will be free to use, and you will be
taught how to use them as part of the class.

\textbf{Course content:} We will cover chapters 12-22 of the
\emph{Organic Chemistry} text. Key topics to be studied include:
understanding how structure determines function and reactivity of
organic molecules. In every topic, we seek a \textbf{conceptual
understanding} from several perspectives, the ability to \textbf{apply
ideas}, development of \textbf{logical reasoning and communication
skills}, and an \textbf{appreciation for organic chemistry as a whole}.

\hypertarget{learning-objectives}{%
\section{Learning objectives}\label{learning-objectives}}

Upon successful completion of this course you will be able to:

\begin{itemize}
\tightlist
\item
  Communicate the structure and properties using drawing and naming
  conventions introduced in CHEM 128A
\item
  Analyze electronic structure of molecules to make and defend
  predictions about their properties.
\item
  Predict products, infer substrates, and propose reagents needed to
  complete a chemical reactions.
\item
  Use curved arrow notation to depict plausible reaction mechanisms.
\item
  Use spectroscopic data to deduce the structure of the molecule.
\end{itemize}

\hypertarget{learning-modules}{%
\subsection{Learning modules}\label{learning-modules}}

\begin{itemize}
\tightlist
\item
  Module 1: Recap of CHEM 128A material
\item
  Module 2: Alcohols, phenols, and thiols (Ch12)
\item
  Module 3: Ethers, epoxides, and sulfides (Ch13)
\item
  Module 4: Infrared spectroscopy and mass spectrometry (Ch14)
\item
  Module 5: NMR spectroscopy (CH15)
\item
  Module 6: Conjugated pi systems and pericyclic reactions (Ch16)
\item
  Module 7: Strucrure and properties of aromatic compounds (Ch17)
\item
  Module 8: Reactions of aromatic compounds (Ch18)
\item
  Module 9: Aldehydes and ketones (Ch19)
\item
  Module 10: Carboxylic acids and their derivatives (Ch20)
\item
  Module 11: Enols and enolates (Ch21)
\item
  Module 12: Amines (Ch22)
\end{itemize}

The work assigned in this course is designed to help you achieve course
learning objectives. Learning is a constructive process, rather then
simple transfer of knowledge. Prior to the class meeting, students are
required to work actively to get their first contact with new concepts
by reading the textbook, watching videos, and completing the pre-class
assignments and homework (WileyPLUS). During the synchronous meetings we
will learn through group problem-solving activities, discussions driven
by classroom response systems, and more. After class meeting, students
should regularly study the material by doing practice problems,
completing online homework, and other assignments.

\hypertarget{types-of-graded-work}{%
\section{Types of graded work}\label{types-of-graded-work}}

There are four types of assignments and tests that you will encounter in
this course:

\begin{enumerate}
\def\labelenumi{\arabic{enumi}.}
\tightlist
\item
  Learning Target Assessments (LTAs).\\
\item
  Application/Extension Problems (AEPs)
\item
  Preparation, practice, and participation (PPP)
\item
  Final Exam.
\end{enumerate}

\hypertarget{ltas-and-aeps}{%
\subsection{LTAs and AEPs}\label{ltas-and-aeps}}

LTAs are short tests assessing student learning within one learning
objective. Throughout the semester, you will be asked to provide
evidence that you mastered the skills and concepts that by completing
short quizzes, each addressing a single Learning Target. LTAs are graded
either \emph{satisfactory} or \emph{progressing} and no partial credit
is awarded. See \protect\hyperlink{grading}{``How work is graded in CHEM
128B''} below for details.

AEPs are more challenging integrated problems for which students must to
clearly communicate a complete solution. AEPs assesses student skills
across multiple learning objectives, may require technology, and all
will require a formal writeup. AEPs are graded using the EMRN rubric
discussed in \protect\hyperlink{grading}{``How work is graded in CHEM
128B''} and can be revised and resubmitted if needed.

\hypertarget{preparation-practice-and-participation-ppp}{%
\subsection{Preparation, practice, and participation
(PPP)}\label{preparation-practice-and-participation-ppp}}

This category includes poins earned on online pre-class assignments and
online homework (WileyPLUS) as well as participation during lecture
(iClicker). It is in your interest to complete pre-class assignments
because the results guide my decisions about what activities to plan and
what concepts to focus on in the upcoming class meeting.

You will also have the opportunity to earn points for participation in
clicker questions. Research shows that classroom response systems
(clickers) help students learn more and do better in the course. I have
successfully used the student response system to gauge student learning
and direct the flow of the lecture. You will earn one point for
participation in polling sessions and one point for each correctly
answered question.

\hypertarget{final-examination}{%
\subsection{Final examination}\label{final-examination}}

Final exam will be on \textbf{Monday, December 14th, 08:45--10:45 AM}.
Final exam is composed of 70 multiple choice questions designed by
experts from the American Chemical Society. The final exam will be
administered online via Canvas and the Respondus Lockdown Browser.

\hypertarget{grading}{%
\section{How work is graded in CHEM 128B}\label{grading}}

CHEM 128B uses a mastery-based grading system that is designed to
provide you with control over the grading process. The final grade in
CHEM 128B will be determined by the quantity and quality of evidence you
provide that show you have mastered the course learning objectives.
There are 24 Learning Targets in the course, 10 of which are designated
as \textbf{Core} targets due to their central nature in Organic
Chemistry, and the other 14 of which are designated as
\textbf{Supplemental}. Students get numerous opportunities to
demonstrate understanding of the Learning Targets; every time this
happens, the student receives a ``check'' on that Learning Target.

\hypertarget{final-letter-grade}{%
\subsection{Final letter grade}\label{final-letter-grade}}

Your course grade is determined using the table below. In order to earn
a particular letter grade, each requirement must be met in the column
for that grade, \textbf{the highest grade for which all the requirements
are met}. There are no statistical or numerical adjustments (a.k.a.
grading on a curve). Failing grade (F) is given if not all the
requirements for a ``D'' are met.

Note: In the table, numerical values indicate the minimum level needed
to meet the requirement; amounts above this level also meet the
requirement. For AEP's, ``M+'' means ``either M or E''.

\begin{longtable}[]{@{}ccccc@{}}
\toprule
Category & D & C & B & A\tabularnewline
\midrule
\endhead
Core Learning Targets (10) & 5 & 10 & 10 & 10\tabularnewline
Supplemental Learning Targets (14) & 5 & 6 & 9 & 12\tabularnewline
AEPs (8+) & 1 M+ & 2 M+ & 2 E, 2 M+ & 3E, 3 M+\tabularnewline
PPP (1000+) & 500 & 750 & 750 & 750\tabularnewline
Final Exam & 20\% & 30\% & 40\% & 50\%\tabularnewline
\bottomrule
\end{longtable}

\hypertarget{revisions}{%
\subsection{Revisions and tokens}\label{revisions}}

The grading system in our course insists that you show consistent
excellence in all assignments in the course---outstanding work on
homework, for example, does not ``bring up'' poor work on LTAs. This can
be challenging, but the course also provides a robust system of revision
and reassessment for most graded tasks, so that if you aren't happy with
a grade on an assignment, you'll have multiple chances to try again or
fix any mistakes.

Scores for WileyPLUS assignments are final. If you do not get a
percentage correct to show mastery (\textgreater75\%), you can reset the
assignment and try again until the deadline.

Students can request reassessment for any unsuccessful LTAs. Each
additional attempt will cover the same material and have similar
problems but will not be identical to past quizzes.

Students can attempt up to three LTAs during a reassessment session or
up to two LTAs during a single office hour visit. The 20-minute of LTA
is firm and no extra time will be allowed.

To request a (re)assessment, student must reserve an appointment on
Canvas (look for ``Find Appointments'' feature in Calendar).

I found that students tend to defer LTAs until it's too late. Thus, if
an attempt at passing a chapter LTA occurs later then 3 weeks after
covering it in class (first attempt) or 3 weeks after previous
unsuccesful attempt, it will cost two
\protect\hyperlink{tokens}{tokens}.

If you receive either a \emph{Progressing} or \emph{Incomplete} mark,
you will receive feedback on your work, and you can use the feedback to
make corrections and then resubmit your work for regrading. You may
submit up to one revision per week.

\hypertarget{tokens}{%
\subsection{Tokens}\label{tokens}}

Tokens are a ``currency'' in the course that you use to purchase LTA
attempts, assignment regrades, and exceptions to some course rules. Each
student begins the course with 20 tokens, and tokens can purchase any of
the following:

\begin{itemize}
\tightlist
\item
  one token buys one attempt at passing one LTA;
\item
  one token buys 24-hour deadline extension for online ``Mastery''
  homework;
\item
  two tokens buy one attempt at one LTA beyond the three week window;
\item
  two tokens buy one participation credit; and
\item
  three tokens buy ORION proficiency meter reset.
\end{itemize}

\hypertarget{what-are-my-expectations}{%
\section{What are my expectations?}\label{what-are-my-expectations}}

\textbf{I want you to be successful in this course.} I will do my utmost
to help you do this, by creating and maintaining a learning environment
based on challenge and support and giving my highest professional
commitment to your success and well-being. But, \textbf{I cannot achieve
success for you}. Success in college courses comes from cooperation with
instructors, interaction with your classmates, and diligent effort
throughout the semester. I like to compare successful classroom
interactions to interactions between players and coaches on a sports
team. Players do the work and coaches make sure players do the work that
helps players succeed.

To be successful in the course, you need to make sure you are always
giving an effort to do the following:

\begin{itemize}
\tightlist
\item
  Prepare for the class through the pre-class learning exercises (Skill
  Builder).
\item
  Attend all class meetings and participate in class activities.
\item
  Be proactive in completing course work and avoid procrastination in
  all things.
\item
  Maintain awareness of course announcements and calendar events, by
  regularly checking email, and the course calendar.
\item
  Take initiative to seek out help when you are stuck or have a question
  by using office visits, SI, study groups, and whatever else works for
  you.
\item
  Maintain a positive attitude about the class and what you are
  learning.
\end{itemize}

There are many strategies to study Organic Chemistry. The hardest and
one I don't recommend is rote memorization. There will be a lot of new
words, definitions, names, and structures that you will have to commit
to memory. Memorizing \emph{everything}, however, is nearly impossible
because the amount of material that is covered increases dramatically as
the semester progresses. Understanding of the trends, principles,
connections, and logic of chemical transformation will give you better
chances of success.

\hypertarget{expectations-for-professor}{%
\subsection{Expectations for
professor}\label{expectations-for-professor}}

My primary responsibility is to create a learning environment where it's
safe to take risks and make mistakes, without shaming or judgment, and
to give you feedback and guidance as you grow in your understanding of
the subject. As my students, you have a right to expect from me:

\begin{itemize}
\tightlist
\item
  carefully designed and executed learning activities both in and out of
  class;
\item
  informative feedback on, and timely return of all graded work (I
  strive to return all graded work within one week of your turning it
  in);
\item
  timely responses to all communications; and
\item
  respectful, professional treatment in all personal interactions.
\end{itemize}

If you perceive that I am falling short in any of these expectations,
you have the right and responsibility to give constructive feedback that
helps me improve. I will consider all reasonable suggestions in the
course regarding my instruction or the course design.

\hypertarget{supplemental-instruction}{%
\subsection{Supplemental Instruction}\label{supplemental-instruction}}

Supplemental Instruction (SI) is provided for all students who want to
improve their understanding of the material taught in this course. SI
sessions are led by a student who has already mastered the course
material and has been trained to facilitate group sessions where
students can meet to compare class notes, review and discussimportant
concepts, develop strategies for studying, and prepare for exams. The SI
leader attends this class and communicates regularly with the instructor
to ensure that accurate information is given. Attendance at SI sessions
is free and voluntary for any student enrolled in this course. Students
may attend as many times as they choose. A session schedule will be
announced in the first few weeks of class. Learn more by watching this
video: \url{http://youtu.be/yTLGu5TLOUI}

\hypertarget{course-policies}{%
\section{Course policies}\label{course-policies}}

\hypertarget{technology-issues-when-submitting-work}{%
\subsection{Technology issues when submitting
work}\label{technology-issues-when-submitting-work}}

WileyPLUS and ORION assignments are submitted electronically. It is the
student's responsibility to make sure these items are submitted on time,
through any means necessary, even if technology issues arise. If a tech
issue arises that prevents your being able to submit work on time, it is
your responsibility to find another way to get it to me (for example,
via an email attachment). Technology issues that are avoidable or
resolved with a simple work-around will not be considered valid grounds
for a deadline extension. For example, if you are trying to upload a Lab
to Canvas and Canvas won't accept the file, you should try again later
or send the file as an email attachment until you can upload it
successfully.

\hypertarget{recording-of-in-class-content}{%
\subsection{Recording of in-class
content}\label{recording-of-in-class-content}}

Audio and video recordings of class lectures are prohibited unless I
give you explicit permission to do it. Students with an official letter
from the Services for Students with Disabilities office may record the
class if SSD has approved that service.

\hypertarget{academic-dishonesty}{%
\subsection{Academic Dishonesty}\label{academic-dishonesty}}

Your work on Learning Target Assessments must be done individually, and
all collaboration is prohibited.

For take-home assignments you are allowed and encouraged to work with
others. However, the final product that you submit for feedback must be
the result of your own efforts. Therefore, you may share ideas and
strategies with others, but collaboration on the actual finished product
you submit is not allowed. Your work is expected to be the product of
your own thinking, written and explained in your own words with no parts
of the work copied from external sources such as books or websites, and
done clearly enough in your own mind that you could explain the work
from start to finish if asked. Specifically, this excludes:

\begin{itemize}
\tightlist
\item
  copying work from another student;
\item
  copying work from a website;
\item
  paraphrasing work done by another student or from print or internet
  resources---i.e.~putting it in your own words---without coming up with
  the main ideas and strategies yourself; and
\item
  \emph{allowing or enabling} another student to copy or paraphrase work
  that you did, even if you did the original work yourself.
\end{itemize}

Violation of this policy is considered ``academic dishonesty'' and
carries with it strong punitive measures mandated by Fresno State,
including possible automatic failure of the course or suspension from
the university. For details, please see APM 235 by going to
\url{http://www.fresnostate.edu/aps/documents/apm/235.pdf}.

You may feel tempted to academic dishonesty at some point in the
semester. The work can be difficult, and many of you are under a lot of
stress. If you are considering academic dishonesty, please STOP, take a
breath, and remember that your classmates and I want you to succeed in
the course. You are not alone, and you have a strong network in the
class for getting help. The revision and resubmission policies mean that
it's OK to turn in work that isn't perfect. There is no need to be
academically dishonest! Just do your best on the work, and you'll have
the chance to revise it later.

\hypertarget{dropping-the-course}{%
\subsection{Dropping the course}\label{dropping-the-course}}

Students may drop classes using the on-line system through Thursday,
February. The Drop/Withdrawal Form, signed by instructor and department
chair, is needed to drop a course after that date. Withdrawals processed
before 9/20 will not show on the official transcript. Serious and
compelling drop period begins on September 21 and ends on November 20.
More details on
\href{http://fresnostate.edu/studentaffairs/are/registration/add-drop-deadlines.html}{Admissions
web pages}

A \emph{serious and compelling reason} is defined as an unexpected
condition that is not present prior to enrollment in the course that
unexpectedly arises and interferes with a student's ability to attend
class meetings and/or complete course requirements. The reason must be
acceptable to and verified by the instructor of record and the
department chair. The condition must be stated in writing on the
appropriate form. The student must provide documentation that
substantiates the condition.

Failing or performing poorly in a class is not an acceptable ``serious
and compelling reason'' within the University policy, nor is
dissatisfaction with the subject matter, class or instructor.

\hypertarget{university-policies-and-disclaimers}{%
\section{University policies and
disclaimers}\label{university-policies-and-disclaimers}}

In addition to course policies, you are expected to be familiar with
Academic Regulations described in the
\href{http://www.fresnostate.edu/catalog/academic-regulations/}{University
Catalog} as well as policies listed below.

\begin{itemize}
\tightlist
\item
  Class Schedule Policies:
  \url{http://fresnostate.edu/studentaffairs/classschedule/policy/}
\item
  Copyright Policy: \url{http://libguides.csufresno.edu/copyright}
\item
  Students with Dissabilities:
  \url{http://fresnostate.edu/studentaffairs/careers/students/interests/disabilities.html}
\item
  Academic Integrity:
  \url{http://fresnostate.edu/studentaffairs/studentconduct/academic-integrity/}
\item
  Policy on Cheating and Plagiarism:
  \url{http://fresnostate.edu/studentaffairs/studentconduct/policies/cheating-plagiarism.html}
\item
  Add/Drop Course:
  \href{http://fresnostate.edu/studentaffairs/classschedule/registration/add-drop.html}{http://www.fresnostate.edu/studentaffairs/registrar/registration/add-drop-deadlines.html}
\end{itemize}

\hypertarget{LT}{%
\section{Appendix B: Course Modules and Learning Targets}\label{LT}}

\hypertarget{learning-modules-1}{%
\subsection{Learning modules}\label{learning-modules-1}}

\begin{itemize}
\tightlist
\item
  Module 1: Recap of CHEM 128A material
\item
  Module 2: Alcohols, phenols, and thiols (Ch12)
\item
  Module 3: Ethers, epoxides, and sulfides (Ch13)
\item
  Module 4: Infrared spectroscopy and mass spectrometry (Ch14)
\item
  Module 5: NMR spectroscopy (CH15)
\item
  Module 6: Conjugated pi systems and pericyclic reactions (Ch16)
\item
  Module 7: Strucrure and properties of aromatic compounds (Ch17)
\item
  Module 8: Reactions of aromatic compounds (Ch18)
\item
  Module 9: Aldehydes and ketones (Ch19)
\item
  Module 10: Carboxylic acids and their derivatives (Ch20)
\item
  Module 11: Enols and enolates (Ch21)
\item
  Module 12: Amines (Ch22)
\end{itemize}

\hypertarget{learning-targets}{%
\subsection{Learning Targets}\label{learning-targets}}

\textbf{Group E}: Analyze electronic structure of molecules to make and
defend predictions about properties and reactivity of organic molecules.

\begin{itemize}
\tightlist
\item
  E1: State and explain acidity of alcohols as Brønsted-Lowry acids
\item
  E2: Use the MO theory to state and explain the origins of selectivity
  in addition reactions of dienes and pericyclic reactions (thermal and
  photochemical)
\item
  E3: \textbf{(CORE)} Construct MO diagram and use the MO theory to
  predict/explain whether a compound is aromatic, antiaromatic, or
  non-aromatic
\item
  E4: Determine the effect of the existing substituents on the rate and
  regioselectivity of electrophilic aromatic substitution
\item
  E5: \textbf{(CORE)} State and explain acidity of carboxylic acids and
  amines (and basicity of their conjugate bases) in light of the
  Brønsted-Lowry theory
\item
  E6: State and explain acidity of enolate precursors as Brønsted-Lowry
  acids
\end{itemize}

\textbf{Group R}: Predict products, infer substrates, and propose
reagents needed to complete a chemical reaction scheme.

\begin{itemize}
\tightlist
\item
  R1: \textbf{(CORE)} Reactions involving alcohols
\item
  R2: Reactions involving ethers and sulfides
\item
  R3: Reactions involving conjugated pi systems
\item
  R4: \textbf{(CORE)} Aromatic substitution reactions
\item
  R5: \textbf{(CORE)} Reactions involving aldehydes, ketones, imines,
  and enamines
\item
  R6: \textbf{(CORE)} Reactions involving carboxylic acids and its
  derivatives
\item
  R7: Reactions involving enols and enolates
\item
  R8: Reactions involving amines
\end{itemize}

\textbf{Group M}: Use curved arrow notation to depict plausible reaction
mechanisms

\begin{itemize}
\tightlist
\item
  M1: Draw mechanisms of reactions that involve alcohols
\item
  M2: Draw a mechanism to predict the outcome of ring-opening reaction
  of epoxides
\item
  M3: \textbf{(CORE)} Draw a mechanism to predict the outcome and
  selectivity of aromatic substitution reaction.
\item
  M5: \textbf{(CORE)} Draw mechanisms of reactions that involve
  aldehydes, ketones, imines, and enamines
\item
  M4: \textbf{(CORE)} Draw mechanisms of reactions that involve
  carboxylic acids and its derivatives
\item
  M6: Draw mechanisms of reactions that involve enols and enolates
\item
  M7: Draw mechanisms of reactions that involve amines
\end{itemize}

\textbf{Group S}: Use spectroscopic data to determine the structure of a
molecule

\begin{itemize}
\tightlist
\item
  S1: \textbf{(CORE)} Use IR spectroscopy to determine what functional
  groups are present in the analyzed sample
\item
  S2: Use MS spectrometry and the degree of unsaturation to deduce
  structural information about the analyzed sample
\item
  S3: Use NMR data and multiplet analysis to deduce detailed structural
  information about the analyzed sample
\end{itemize}
